\documentclass[10pt,a4paper]{article}
\usepackage[utf8]{inputenc}
\usepackage{amsmath}
\usepackage{amsfonts}
\usepackage{amssymb}
\usepackage{amsthm}
\usepackage{float}
\usepackage{mathtools}
\usepackage{geometry}[margin=1in]
\usepackage{xspace}
\usepackage{tikz}
\usepackage{mathrsfs}
\usetikzlibrary{shapes, arrows, decorations.pathmorphing, ducks, automata}
\usepackage[parfill]{parskip}
\usepackage{subcaption}
\usepackage{stmaryrd}
\usepackage{marvosym}
\usepackage{dsfont}
\usepackage{pgfplots}
\usepackage{enumitem}
\usepackage{calc}
\usepackage{tikz-cd}
\usepackage{hyperref}

\hypersetup{
    colorlinks,
    citecolor=black,
    filecolor=black,
    linkcolor=black,
    urlcolor=black
}

\newcommand{\st}{\text{ s.t. }}
\newcommand{\contr}{\lightning}
\newcommand{\im}{\mathfrak{i}}
\newcommand{\R}{\mathbb{R}}
\newcommand{\Q}{\mathbb{Q}}
\newcommand{\C}{\mathbb{C}}
\newcommand{\F}{\mathbb{F}}
\newcommand{\K}{\mathbb{K}}
\newcommand{\N}{\mathbb{N}}
\newcommand{\Z}{\mathbb{Z}}
\renewcommand{\P}{\mathbb{P}}
\renewcommand{\H}{\mathds{H}}
\renewcommand{\O}{\mathcal{O}}
\newcommand{\A}{\mathbb{A}}
\newcommand{\D}{\mathbb{D}}
\newcommand{\nequiv}{\not\equiv}
\newcommand{\powset}{\mathcal{P}}
\renewcommand{\th}[1][th]{\textsuperscript{#1}\xspace}
\newcommand{\from}{\leftarrow}
\newcommand{\legendre}[2]{\left(\frac{#1}{#2}\right)}
\newcommand{\ow}{\text{otherwise}}
\newcommand{\imp}[2]{\underline{\textit{#1.}$\implies$\textit{#2.}}}
\let\oldexists\exists
\let\oldforall\forall
\renewcommand{\exists}{\oldexists\;}
\renewcommand{\forall}{\;\oldforall}
\renewcommand{\hat}{\widehat}
\renewcommand{\tilde}{\widetilde}
\newcommand{\one}{\mathds{1}}
\newcommand{\under}{\backslash}
\newcommand{\injection}{\hookrightarrow}
\newcommand{\surjection}{\twoheadrightarrow}
\newcommand{\jacobi}{\legendre}
\newcommand{\floor}[1]{\lfloor #1 \rfloor}
\newcommand{\ceil}[1]{\lceil #1 \rceil}
\newcommand{\cbrt}[1]{\sqrt[3]{#1}}
\renewcommand{\angle}[1]{\langle #1 \rangle}
\newcommand{\dbangle}[1]{\angle{\angle{#1}}}
\newcommand{\wrt}{\text{ w.r.t. }}

\newcommand*\circled[1]{\tikz[baseline=(char.base)]{
      \node[shape=circle,draw,inner sep=2pt] (char) {#1};}
}

\DeclareMathOperator{\ex}{ex}
\DeclareMathOperator{\id}{id}
\DeclareMathOperator{\upper}{Upper}
\DeclareMathOperator{\dom}{dom}
\DeclareMathOperator{\disc}{disc}
\DeclareMathOperator{\charr}{char}
\DeclareMathOperator{\Image}{im}
\DeclareMathOperator{\ord}{ord}
\DeclareMathOperator{\lcm}{lcm}
\DeclareMathOperator{\aut}{Aut}
\DeclareMathOperator{\diag}{diag}
\DeclareMathOperator{\stab}{stab}
\DeclareMathOperator{\trace}{trace}
\DeclareMathOperator{\ecl}{ecl}
\DeclareMathOperator{\Span}{Span}
\DeclareMathOperator{\Gal}{Gal}
\DeclareMathOperator{\Aut}{Aut}
\DeclareMathOperator{\Frob}{Frob}
\let\div\relax
\DeclareMathOperator{\div}{div}
\DeclareMathOperator{\Div}{Div}
\let\Re\relax
\let\Im\relax
\DeclareMathOperator{\Re}{\mathfrak{Re}}
\DeclareMathOperator{\Im}{\mathfrak{Im}}
\DeclareMathOperator{\Frac}{Frac}
\DeclareMathOperator{\Pic}{Pic}

\let\emph\relax
\DeclareTextFontCommand{\emph}{\bfseries\em}

\newtheorem{theorem}{Theorem}[section]
\newtheorem{lemma}[theorem]{Lemma}
\newtheorem{corollary}[theorem]{Corollary}
\newtheorem{proposition}[theorem]{Proposition}
\newtheorem{conjecture}[theorem]{Conjecture}
\newtheorem{definition}[theorem]{Definition}

\definecolor{burgundy}{rgb}{0.5, 0.0, 0.13}

\tikzset{sketch/.style={decorate,
 decoration={random steps, amplitude=1pt, segment length=5pt},
 line join=round, draw=black!80, very thick, fill=#1
}}


\title{Local Fields}
\begin{document}
\maketitle

\section{Basic Theory}
Suppose we have a diophantine polynomial $f(x_1, \ldots, x_r) \in \Z[x_1, \ldots, x_r]$. Then we might want to find integer solutions to the equation $f(x_1, \ldots, x_r) = 0$. However, it turns out this can be very difficult to do, for instance showing $x^n + y^n - z^n = 0$ has no solutions for $x,y,z \in \Z$ took hundreds of years and a lot of advanced mathematics.

Instead, we study congruences of the form $f(x_1, \ldots, x_r) \equiv 0 \mod p^n$, for prime $p$ and integer $n$. This then becomes a finite computation, and hence a much easier problem. Local fields will give us a way to package all this information together.

\subsection{Absolute Values}
\begin{definition}
Let $K$ be a field. An \emph{absolute value} on $K$ is a function $|\cdot|:K \to \R_{\geq 0}$ such that:
\begin{enumerate}
  \item $|x| = 0 \iff x = 0$
  \item $|xy| = |x||y|\forall x, y \in K$
  \item $|x+y| \leq |x|+|y|\forall x, y \in K$
\end{enumerate}
We say that $(K, |\cdot|)$ is a valued field.
\end{definition}
\underline{Examples: }
\begin{enumerate}
  \item $K = \R$ or $\C$ with $|\cdot|$ the usual absolute value. We write $|\cdot|_\infty$ for this absolute value.
  \item $K$ is any field. The \emph{trivial absolute value} on $K$ is defined by:
  \begin{align}
    |x| = \begin{cases} 0 & x = 0 \\ 1 & x \neq 0 \end{cases}
  \end{align}
  We will ignore this absolute value in this course.
  \item $K = \Q$, $p$ a prime. For $0 \neq x \in \Q$, we can write $x = p^n \frac{a}{b}$, where $a, b \in \Z, (a, p) = 1$, and $(b, p) = 1$. The \emph{p-adic absolute value} is defined to be:
  \begin{align*}
    |x|_p = \begin{cases} 0 & x = 0 \\ p^{-n} & x = p^n\frac{a}{b} \end{cases}
  \end{align*}
  We check the axioms.
  \begin{enumerate}[label=\textit{\arabic*.}]
    \item Clear from the definition.
    \item $|xy|_p = |p^{m+n}\frac{ac}{bd}|_p = p^{-m-n} = |x|_p|y|_p$
    \item \textsc{Wlog}, $m\geq n$. $|x+y|_p = \left|p^n\left(\frac{ad+p^{m-n}bc}{bd}\right)\right|_p \leq p^{-n} = \max(|x|_p, |y|_p)$
  \end{enumerate}
\end{enumerate}
An absolute value on $K$ induces a metric $d(x,y) = |x-y|$ on $K$, and hence induces a topology on $K$. As an exercise, check that $+, \cdot$ are continuous.

\begin{definition}
  Let $|\cdot|, |\cdot|'$ be absolute values on a field $K$. We say that $|\cdot|, |\cdot|'$ are \emph{equivalent} if they induce the same topology on $K$. An equivalence class of absolute values is called a \emph{place}.
\end{definition}

\begin{proposition}
  Let $|\cdot|, |\cdot|'$ be non-trivial absolute values on $K$. The following are equivalent:
  \begin{enumerate}
    \item $|\cdot|, |\cdot|'$ are equivalent.
    \item $|x| < 1 \iff |x|' < 1 \forall x\in K$.
    \item $\exists c \in \R_{>0} \st |x|^c = |x|' \forall x \in K$
  \end{enumerate}
\end{proposition}
\begin{proof}\hspace*{0cm}\\
  \imp{1}{2}
  \begin{align}
    |x|<1 &\iff x^n \to 0 \wrt |\cdot|\\
    &\iff x^n \to 0 \wrt |\cdot|'\\
    &\iff |x|' < 1
  \end{align}

  \imp{2}{3} Let $a \in K^{\times} \st |a| < 1$, which exists since $|\cdot|$ is non-trivial. We need to show that, for all $x \in K^\times$, we have:
  \begin{align*}
    \frac{\log|x|}{\log|a|} = \frac{\log|x|'}{\log|a|'}
  \end{align*}
  Assume $\frac{\log|x|}{\log|a|} < \frac{\log|x|'}{\log|a|'}$. Then choose $m,n \in \Z$ so that $\frac{\log|x|}{\log|a|} < \frac{m}{n} < \frac{\log|x|'}{\log|a|'}$. Then we have:
  \begin{align*}
    n \log |x| &< m \log |a|\\
    n \log |x|' &> m \log |a|'
  \end{align*}
  and hence $|\frac{x^n}{a^m}| < 1, |\frac{x^n}{a^m}|' > 1, \contr.$

  \imp{3}{1} This is clear, as open balls in one topology will also be open balls in the other, hence the topologies will be the same.
\end{proof}

In this course, we will be mainly interested in the following types of absolute values:
\begin{definition}
  An absolute value $|\cdot|$ on $K$ is said to be \emph{non-archimedean} if it satisfies the ultrametric inequality $|x+y| \leq \max(|x|, |y|)$
\end{definition}
If $|\cdot|$ is not non-archimedean, then it is archimedean.\\
\underline{Examples: }
\begin{enumerate}
  \item $|\cdot|_\infty$ on $\R$ is archimedean.
  \item $|\cdot|_p$ is a non-archimedean absolute value on $\Q$.
\end{enumerate}
\begin{lemma}[All triangles are isosceles]
    Let $(K, |\cdot|)$ be a non-archimedean valued field, and $x, y \in K$. If $|x|<|y|$, then $|x-y| = |y|$.
\end{lemma}
\begin{proof}
  Observe that $|1| = |1\cdot 1| = |1|\cdot |1|$, and so $|1| = 1$ or $0$. But $1 \neq 0$, so $|1| = 1$. Similarly, $|-1| = 1$, and so $|-y| = |y|$ for all $y \in K$.

  Then if $|x| < |y|$,  $|x-y| \leq \max(|x|, |y|) = |y|$.

  At the same time $|y| \leq \max(|x|, |x-y|) \implies |y| \leq |x-y|$.

  Hence $|y| = |x-y|$.
\end{proof}

\begin{proposition}
  Let $(K, |\cdot|)$ be non-archimedean, and $(x_n)_{n=1}^\infty$ be a sequence in $K$.

  If $|x_n-x_{n+1}| \to 0$, then $(x_n)_{n=1}^\infty$ is Cauchy.

  In particular, if $K$ is in addition complete, then $(x_n)_{n=1}^\infty$ converges.
\end{proposition}
\begin{proof}
  For $\epsilon > 0$, choose $N$ such that $|x_n - x_{n+1}| < \epsilon\forall n>N$.

  Then for $N < n < m$, we have:
  \begin{align*}
    |x_n-x_m| &= |(x_n-x_{n+1}) + (x_{n+1}-x_{n+1}) +\ldots + (x_{m-1} - x_m)| < \epsilon
  \end{align*}
  And so the sequence is Cauchy.
\end{proof}

For example, if $p=5$, construct the sequence $(x_n)_{n=1}^\infty$ such that:
\begin{enumerate}
  \item $x_n^2+ 1\equiv 0 \mod 5^n$
  \item $x_n \equiv x_{n+1} \mod 5^n$
\end{enumerate}
as follows:

Take $x_1 = 2$. Suppose we have constructed $x_n$. Let $x_n^2 + 1 = a5^n$, and set $x_{n+1} = x_n + b5^n$. Then $x_{n+1}^2 + 1=  x_n^2 + 2b5^nx_n + b^2 5^{2n} + 1 = a5^n + 2b5^nx_n + b^2 5^{2n}$.

We choose $b$ such that $a+2bx_n \equiv 0 \mod 5$, i.e. $b \equiv -\frac{a}{2x_n} \mod 5$, and then we have $x_{n+1}^2 +1 \equiv 0 \mod 5^{n+1}$ as desired.

The second property implies that $|x_{n+1} - x_n|_5 < 5^{-n} \to 0$, and so the sequence is Cauchy. Now suppose that $x_n \to L \in \Q$. Then $x_n^2 \to L^2$. But the first property then gives us that $x_n^2 \to -1 \implies L^2 = -1 \contr.$ So $(\Q, |\cdot|_5)$ is not complete.

\begin{definition}
  The $p$-adic numbers $\Q_p$ is the completion of $\Q$ with respect to $|\cdot|_p$.
\end{definition}
We have an analogy with $\R$, in that $\R$ is the completion of $\Q$ with respect to $|\cdot|_\infty$.
\end{document}
