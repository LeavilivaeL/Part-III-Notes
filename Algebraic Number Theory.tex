\documentclass[10pt,a4paper]{article}
\author{Harry Armitage}

%\usepackage[utf8]{inputenc}
\usepackage{amsmath}
\usepackage{amsfonts}
\usepackage{amssymb}
\usepackage{amsthm}
\usepackage{float}
\usepackage{mathtools}
\usepackage{geometry}[margin=1in]
\usepackage{xspace}
\usepackage{tikz}
\usepackage{mathrsfs}
\usetikzlibrary{shapes, arrows, decorations.pathmorphing, ducks, automata}
\usepackage[parfill]{parskip}
\usepackage{subcaption}
\usepackage{stmaryrd}
\usepackage{marvosym}
\usepackage{dsfont}
\usepackage{pgfplots}
\usepackage{enumitem}
\usepackage{calc}
\usepackage{tikz-cd}
\usepackage{hyperref}
\usepackage[usestackEOL]{stackengine}

\usepackage{fontspec}
\usepackage{newpxtext, newpxmath}
\usepackage{anyfontsize}

\hypersetup{
    colorlinks,
    citecolor=black,
    filecolor=black,
    linkcolor=black,
    urlcolor=black
}

\newcommand{\f}[1]{\mathfrak{#1}}
\newcommand{\p}{\f{p}}

\newcommand{\st}{\text{ s.t. }}
\newcommand{\contr}{\lightning}
\newcommand{\im}{\mathfrak{i}}
\newcommand{\R}{\mathbb{R}}
\newcommand{\Q}{\mathbb{Q}}
\renewcommand{\C}{\mathbb{C}}
\newcommand{\F}{\mathbb{F}}
\newcommand{\K}{\mathbb{K}}
\newcommand{\N}{\mathbb{N}}
\newcommand{\Z}{\mathbb{Z}}
\renewcommand{\P}{\mathbb{P}}
\renewcommand{\H}{\mathds{H}}
\renewcommand{\O}{\mathcal{O}}
\newcommand{\A}{\mathbb{A}}
\newcommand{\D}{\mathbb{D}}
\renewcommand{\G}{\mathbb{G}}
%\newcommand{\nequiv}{\not\equiv}
\newcommand{\powset}{\mathcal{P}}
\renewcommand{\th}[1][th]{\textsuperscript{#1}\xspace}
\newcommand{\from}{\leftarrow}
\newcommand{\legendre}[2]{\left(\frac{#1}{#2}\right)}
\newcommand{\ow}{\text{otherwise}}
\newcommand{\imp}[2]{\underline{\textit{#1.}$\implies$\textit{#2.}}}
\let\oldexists\exists
\let\oldforall\forall
\renewcommand{\exists}{\oldexists\;}
\renewcommand{\forall}{\;\oldforall}
\renewcommand{\hat}{\widehat}
\renewcommand{\tilde}{\widetilde}
\newcommand{\one}{\mathds{1}}
\newcommand{\under}{\backslash}
\newcommand{\injection}{\hookrightarrow}
\newcommand{\surjection}{\twoheadrightarrow}
\newcommand{\isomarrow}{\mathrel{\setstackgap{S}{-0.5pt}\ensurestackMath{\Shortstack{\scriptstyle\sim\\ \longrightarrow}}}}
\newcommand{\jacobi}{\legendre}
\newcommand{\floor}[1]{\lfloor #1 \rfloor}
\newcommand{\ceil}[1]{\lceil #1 \rceil}
\newcommand{\cbrt}[1]{\sqrt[3]{#1}}
\renewcommand{\angle}[1]{\langle #1 \rangle}
\newcommand{\dbangle}[1]{\angle{\angle{#1}}}
\newcommand{\wrt}{\text{ w.r.t. }}
\newcommand{\abs}[1]{\lvert#1\rvert}
\newcommand{\norm}[1]{\lVert#1\rVert}
\newcommand*\circled[1]{\tikz[baseline=(char.base)]{
      \node[shape=circle,draw,inner sep=2pt] (char) {#1};}
}
\renewcommand{\epsilon}{\varepsilon}
\newcommand{\trianglerightneq}{\mathrel{\ooalign{\raisebox{-0.5ex}{\reflectbox{\rotatebox{90}{$\nshortmid$}}}\cr$\triangleright$\cr}\mkern-3mu}}
\newcommand{\triangleleftneq}{\mathrel{\reflectbox{$\trianglerightneq$}}}

\DeclareMathOperator{\ex}{ex}
\DeclareMathOperator{\id}{id}
\DeclareMathOperator{\upper}{Upper}
\DeclareMathOperator{\dom}{dom}
\DeclareMathOperator{\disc}{disc}
\DeclareMathOperator{\charr}{char}
\DeclareMathOperator{\Image}{im}
\DeclareMathOperator{\ord}{ord}
\DeclareMathOperator{\lcm}{lcm}
\DeclareMathOperator{\aut}{Aut}
\DeclareMathOperator{\diag}{diag}
\DeclareMathOperator{\stab}{stab}
\DeclareMathOperator{\trace}{trace}
\DeclareMathOperator{\ecl}{ecl}
\DeclareMathOperator{\Span}{Span}
\DeclareMathOperator{\Gal}{Gal}
\DeclareMathOperator{\Aut}{Aut}
\DeclareMathOperator{\Frob}{Frob}
\DeclareMathOperator{\Det}{Det}
\let\div\relax
\DeclareMathOperator{\div}{div}
\DeclareMathOperator{\Div}{Div}
\let\Re\relax
\let\Im\relax
\DeclareMathOperator{\Re}{\mathfrak{Re}}
\DeclareMathOperator{\Im}{\mathfrak{Im}}
\DeclareMathOperator{\Frac}{Frac}
\DeclareMathOperator{\Pic}{Pic}
\DeclareMathOperator{\ann}{ann}
\DeclareMathOperator{\Ass}{Ass}
\DeclareMathOperator{\intt}{int}
\DeclareMathOperator{\Hom}{Hom}
\DeclareMathOperator{\End}{End}
\DeclareMathOperator{\tr}{tr}
\DeclareMathOperator{\Tr}{Tr}
\DeclareMathOperator{\Spec}{Spec}
\DeclareMathOperator{\height}{ht}
\DeclareMathOperator{\rank}{rank}
\DeclareMathOperator{\Art}{Art}
\DeclareMathOperator{\gr}{gr}
\DeclareMathOperator{\Tor}{Tor}
\DeclareMathOperator{\Ext}{Ext}
\DeclareMathOperator{\coker}{coker}

\let\emph\relax
\DeclareTextFontCommand{\emph}{\bfseries\em}

\newtheorem{theorem}{Theorem}[section]
\newtheorem{lemma}[theorem]{Lemma}
\newtheorem{corollary}[theorem]{Corollary}
\newtheorem{proposition}[theorem]{Proposition}
\newtheorem{conjecture}[theorem]{Conjecture}
\newtheorem{definition}[theorem]{Definition}

\definecolor{burgundy}{rgb}{0.5, 0.0, 0.13}

\tikzset{sketch/.style={decorate,
 decoration={random steps, amplitude=1pt, segment length=5pt},
 line join=round, draw=black!80, very thick, fill=#1
}}


\title{Algebraic Number Theory}
\begin{document}
\maketitle
\tableofcontents
\newpage
\section{Absolute Values and Places}
$K$ is a field. An \emph{absolute value} (AV) on $K$ is a function $\abs{\cdot}:K \to \R_{\geq 0}$ such that, for all $x, y \in K$:
\begin{enumerate}[label=\roman*)]
  \item $|x| = 0 \iff x= 0$
  \item $|xy| = |x||y|$
  \item $|x+y| \leq |x|+|y|$
\end{enumerate}
We will also assume that $\abs{\cdot}$ is not trivial, i.e.
\begin{enumerate}
  \item[iv)] $\exists x \in K : |x|\neq 0,1$
\end{enumerate}
An AV is \emph{non-archimedean} if it satisfies
\begin{enumerate}
  \item[iii-NA)] $|x+y| \leq \max(|x|, |y|)$
\end{enumerate}
and \emph{archimedean} otherwise.

An AV determines a metric $d(x,y) = |x-y|$ which makes $K$ a \emph{topological field}.

\textbf{Remark} It's convenient to weaken iii):
\begin{enumerate}
  \item[iii')] $\exists \alpha > 0 \st \forall x,y, |x+y|^{\alpha} \leq |x|^\alpha+|y|^\alpha$
\end{enumerate}
For non-archimedean AVs, this makes no difference. What this does mean is that if $\abs{\cdot}$ is an AV, then so is $\abs{\cdot}^{\alpha}$ for any $\alpha > 0$. The point of this is that we want $z \mapsto z\bar{z}$ on $\C$ to be an AV - we'll see why later.

Let's suppose $\abs{\cdot}$ is a non-archimedean AV. Then $\{x\in K : |x|\leq 1\} = R$ is a subring of $K$. It is a local ring with unique maximal ideal $\{|x|<1\} = \mathfrak{m}_R$.

It is a \emph{valuation ring} of $K$ (i.e. $x \in K \setminus R \implies x^{-1}\in R$).
\begin{lemma}
  $R$ is a maximal subring of $K$.
\end{lemma}
\begin{proof}
  Let $x \in K\setminus R$, so $|x|>1$. Then if $y \in K$, there is some $n \geq 0$ with $|yx^{-n}| = \frac{|y|}{|x|^n} \leq 1$. So $y \in x^n R$ for $n \gg 0$, and hence $R[x] = K$. Hence $R$ is maximal.
\end{proof}
There is a general notion of valuation (not nec. $\R$-valued). In the more general context, these valuations are called \emph{rank 1 valuations}, and they have this maximality property.

We say two absolute values $\abs{\cdot}$ and $\abs{\cdot}'$ are \emph{equivalent} if there is $\alpha > 0$ with $\abs{\cdot}' =\abs{\cdot}^{\alpha}$. This is an equivalence relation.
\begin{proposition}
  The following are equivalent:
  \begin{enumerate}[label=\roman*)]
    \item $\abs{\cdot}, \abs{\cdot}'$ are equivalent.
    \item $|x| \leq |y| \iff |x|'\leq |y|'$.
    \item $|x| < |y| \iff |x|' < |y|'$.
  \end{enumerate}
\end{proposition}
\begin{proof}
  From local fields, or exercise.
\end{proof}
\begin{corollary}
  Let $\abs{\cdot}, \abs{\cdot}'$ be non-archimedean AVs, with valuation rings $R, R'$. Then $\abs{\cdot}, \abs{\cdot}'$ are equivalent if and only if $R = R'$ if and only if $R \subset R'$.
\end{corollary}
Equivalent AVs define equivalent metrics, hence the same topologies, hence the \emph{completion} of $K$ with respect to $\abs{\cdot}$ depends only on the equivalence class of $\abs{\cdot}$.

Inequivalent AVs determine ``independent'' topologies in the following sense:
\begin{proposition}[Weak Approximation]
  Let $\abs{\cdot}_i$ for $1\leq i\leq n$ be pairwise inequivalent AVs on $K$, and $a_1, \ldots, a_n \in K$, $\delta > 0$.

  Then there is $x \in K$ such that, for all $i$, $|x-a_i|_i < \delta$.
\end{proposition}
\begin{proof}
  Suppose $z_j \in K$ such that $|z_j|_j > 1$, and $|z_j|_i < 1$ for all $i \neq j$. Then $\abs{\frac{z_j^N}{z_j^N + 1}}_i \to 0$ as $N \to \infty$ if $i \neq j$, and to $1$ if $i=j$.

  So then $x = \sum a_j \frac{z_j^N}{z_j^N + 1}$ works for $N$ sufficiently large. So it's enough to find $z_j$, and by symmetry take $j=1$. We then induct on $n$. The case $n=1$ is trivial.

  Suppose we have $y$ with $|y|_1 > 1$, and $|y|_2, \ldots, |y|_{n-1} < 1$. If $|y|_n < 1$, we're finished, otherwise pick $w \in K$ with $|w|_1 > 1 > |w|_n$, by \textbf{1.2}. If $|y|_n=1$, then $z = y^Nw$ works, and if $|y|_n > 1$, then $z = \frac{y^Nw}{y^N+1}$ works.
\end{proof}
\textbf{Remark.} If $K = \Q$, $\abs{\cdot}_1, \ldots, \abs{\cdot}_n$ are the $p_i$-adic AVs for distinct primes $p_i$ and $a_i \in \Z$, then weak approximation says that, for all $n_i \geq 1$, there is $x \in \Q$ which is a $p_i$-adic integer for all $i$, and $x \equiv a_i \mod p_i^{n_i}$ for all $i$. This is weaker than CRT, which guarantees $x \in \Z$.

\textbf{Definition.}  A \emph{place} of $K$ is an equivalence class of AVs on $K$.

\textbf{Example} $K=\Q$. \emph{Ostrowski's Theorem} implies every AV on $\Q$ is equivalent to one of $\abs{\cdot}_p, \abs{\cdot}_\infty$. So places of $\Q$ are the primes, and $\infty$. We write $V_K$ for the set of places of $K$.

We write $V_{K, \infty}$ for the places given by archimedean AVs (the infinite places).

We write $V_{K,f}$ for the places given by non-archimedean AVs (the finite places).

We often use letters $v, w$ denote places. Given $v \in V_K, K_v$ will denote the completion of $K$ at $v$. If $v:K^\times \to \R$ is a \emph{valuation}, we will also use $v$ to denote the corresponding place, i.e. the equivalence class of AVs $x \mapsto \gamma^{-v(x)}$.

We can restate the weak approximation in terms of places:
\addtocounter{theorem}{-1}
\begin{proposition}
  Let $v_1, \ldots, v_n$ be distinct places of $K$. Then the image of the diagonal inclusion
  \[ K \injection \prod_{1\leq i\leq n}K_{v_i}\]
  is dense.
\end{proposition}
\subsection{Extensions and Places}
Let $L/K$ be finite and separable, and let $v, w$ be places of $K, L$ respectively. Say $w$ \emph{lies over} or \emph{divides} $v$ (notation $w|v$) if $v$ is the restriction of $w$ to $K$.

Then there is a unique continuous $K_v\injection L_w$ extending $K \injection L$.
\begin{proposition}
  There is a unique isomorphism of topological rings
  \[L \otimes_K K_v \xrightarrow{\sim} \prod_{w|v}L_w\]
  mapping $x \otimes y$ to $(xy)_w$.
\end{proposition}
\begin{proof}
  Both sides are finite dimensional normed $K_v$-vector spaces. The idea will be to choose a basis of $L/K$ so that $L \otimes_K K_v \cong K_v^{[L:K]}$ (with the $\sup$ norm), and on the RHS we also use the $\sup$ norm. Then we use the fact that any 2 norms on a finite dimensional vector space over a field complete with respect to an absolute value are equivalent (see Cassesl and Fr\"ohlich, Ch. III, section 8).

  Write $L = K(a)$ where $f \in K[T]$ is a minimal polynomial for $a$, and is separable. Factor $f = \prod g_i$ in $K_v$, so that the $g_i \in K_v[T]$ are irreducible and distinct.

  Let $L_i = K_v[T]/(g_i)$. Then $L \otimes_K K_v = K_v[T]/(f) \cong \prod_i L_i$. Now let $w|v$, inducing $i_w:L \injection I_w$. Let $g_w \in K_v[T]$ be the minimal polynomial of $i_w(a)$ over $K_v$. Then $g_w|f$, so $g_w$ is one of the $g_i$s, and $L_w = K_v(i_w(a)) = L_i$.

  Conversely, $K_v$ is complete and $L_i/K_v$ is finite, so there is a unique extension of $v$ to $L_i$, and we get a bijection $\{g_i\}\leftrightarrow \{w|v\}$, and $L\otimes_K K_v \cong \prod L_w$.
\end{proof}
\begin{corollary}\hspace*{0cm}
  \begin{enumerate}
    \item $\{w|v\}$ is finite, nonempty, and $[L:K] = \sum_{w|v} [L_w:K_v]$
    \item $\forall x \in K$, \\$N_{L/K}(x) = \prod_{w|v}N_{L_w/K_v}(x)$\\
    $\Tr_{L/K}(x) = \sum_{w|v}\Tr_{L_w/K_v}(x)$.
  \end{enumerate}
\end{corollary}
If $L/K$ is Galois with Galois group $G$, then $G$ acts on the places $w$ of $L$ lying over a given $v$: if $\abs{\cdot}$ is an AV on $L$ then, for all $g \in G$, the map $x\mapsto \abs{g^{-1}x}$ is an AV on $L$, agreeing with $\abs{\cdot}$ on $K$, and hence gives a (left) action of $G$ on $\{w|v\}$, $g(w) = w \circ g^{-1}$. If $w=v_P$ for a prime $P$, then $gw = v_{g(P)}$.a

We define the \emph{decomposition group} $D_w$ or $G_w$ to be the stabiliser of $w$. This action is transitive. If $g \in D_w$, then it is continuous for the topology induced by $w$ on $L$, so it extends to an automorphism of $L_w$, the completion of $L$ at $w$.
\[G \supset\Gal(L_w/K_v) \supseteq D_w\]
Then we have
\begin{align*}
  \#G &= (G:G_w)\#G_w\\
  &\leq (G:G_w)[L_w:K_v]\\
  &= \sum_{g \in G/G_w} [L_{g(w)}:K_v]\\
  &\leq \sum_{w|v}[L_w:K_v]\\
  &= [L:K] = \#G
\end{align*}
Hence equality holds throughout, and $G_w = \Gal(L_w/K_v)$.

Suppose $v$ is a \emph{discrete valuation} of $L$, i.e. it is a finite place, and the valuation ring is a DVR. Then so is any $w|v$, and we define:
\begin{itemize}
  \item $f(w|v)$, the degree of residue class extension, $= e_{L_w/K_v}$
  \item $e(w|v)$, the ramification degree
\end{itemize}
and $[L_w:K_v] = e(w|v)f(w|v)$.
\section{Number Fields}
A lot of this theory applies to other global fields, e.g. function fields. $K$ will here be a number field (i.e. finite extension of the rationals) with ring of integers $O_K$. We have some basic properties:
\begin{itemize}
  \item $O_K$ is a \emph{Dedekind domain}, i.e.
  \begin{enumerate}
    \item Noetherian (in fact, $O_K$ is a f.g. $\Z$-module).
    \item Integrally closed in $K$ (by definition).
    \item Every non-zero prime ideal is maximal, so has Krull dimension $\leq 1$.
  \end{enumerate}
\end{itemize}
We have some basic results about Dedekind domains:
\begin{theorem}\hspace*{0cm}
  \begin{enumerate}
    \item A local domain is Dedekind if and only if it is a DVR.
    \item For a domain $R$, TFAE:
    \begin{enumerate}
      \item $R$ is Dedekind.
      \item $R$ is Noetherian and for every non-zero prime $\p$, $R_\p$ is a DVR.
      \item Every fractional ideal of $R$ is invertible.
    \end{enumerate}
    \item A Dedekind domain with only finitely many prime ideals (i.e. \emph{semi-local}) is a PID.
  \end{enumerate}
\end{theorem}
\begin{proof}\hspace*{0cm}
  \begin{enumerate}
    \item Proved in local fields, $\implies$ is the hardest part.
    \item Let $K = \Frac(R)$. A fractional ideal of $R$ is a non-zero $R$-submodule $I \subset K$ for some $0 \neq x \in R$ where $xI \subset R$ is an ideal. For $(a)\implies (b)$ it is enough to check (exercise) that the basic properties are preserved under localisation.

    For $(b)\implies (c)$, $I$ is invertible if there is a fractional ideal $I^{-1}$ such that $II^{-1} = R$. To prove $(c)$, we may assume $I \subset R$ is an ideal. Then let $I^{-1} = \{x \in K : xI \subset R\}$. If $0 \neq y \in I$, then $R \subset I^{-1} \subset y^{-1}R$, and so $I^{-1}$ is a fractional ideal. Clearly $I^{-1}I \subset R$. Now let $P \subset R$ be prime - it is sufficient to show $I^{-1}I \nsubset P$. Let $I = (a_1, \ldots, a_n)$. WLOG take $v_P(a_1) \leq v_P(a_i)$ for all $i >1$. Then $IR_P = a_1R_P$, as $R_P$ is a DVR.

    Hence $a_i/a_1 = x_i/y_i \in R_P$ where $x_i \in R, y_i \in R\setminus P$. Then $y = \prod y_i \notin P$ as $P$ is prime, and $ya_i/a_1 \in R$ for all $i$, and so $y/a_i \in I^{-1}$, so $y \in II^{-1}\setminus P$.

    For $(c)\implies (a)$, we check the properties. $R$ is Noetherian - let $I \subset R$ be an ideal. Then $II^{-1} =R \implies 1 = \sum_{i=1}^n a_ib_i, a_i \in I, b_i \in I^{-1}$. Let $I' = (a_1, \ldots, a_r) \subset I$. Then $I'I^{-1} = R = II^{-1}$, and so $I' = I$, and $I$ is finitely generated.

    $R$ is integrally closed. Let $x \in K$, integral over $R$. Then $I \coloneqq R[x] = \sum_{0\leq i < d} Rx^i \subset K$ is a fractional ideal. Obviously $I^2 = I$, so $I = I^2I^{-1} = II^{-1} = R$, i.e. $x \in R$.

    Every non-zero prime is maximal. Take $\{0\} \neq Q \subset P \subsetneq R$ where $P,Q$ are prime. Then $R \subsetneq P^{-1} \subset Q{^-1}$, and $Q\subsetneq P^{-1}Q \subset R$, and $P(P^{-1}Q) = Q$, so as $Q$ is prime and $P^{-1}Q \nsubset R$, we must have $P\subset Q$, and so $P= Q$.

    \item Let $R$ be a semi-local Dedekind domain with non-zero primes $P_1, \ldots, P_n$. Choose $x \in R$ with $x \in P_1 \setminus P_1^2, x\in P_2, \ldots, P_n$. Then $P_1 = (x)$ and every ideal is a product of powers of $\{P_i\}$ (see below), hence $R$ is a PID.
  \end{enumerate}
\end{proof}
\begin{theorem}
  Let $R$ be Dedekind. Then:
  \begin{enumerate}
    \item The group of fractional ideals is freely generated by the non-zero prime ideals, and
    \[I = \prod_P P^{v_P(I)}\]
    with $v_P(I) = \inf_{x\in I}(v_P(x))$.
    \item If $(R:I) < \infty$ for all $I \neq (0)$, then for all $I, J$, $(R: IJ) = (R:I)(R:J)$.
  \end{enumerate}
\end{theorem}
\begin{proof}\hspace*{0cm}
  \begin{enumerate}
    \item If $I \neq R$, then $I \subset P$ for some prime ideal $P$. Then $I = PI', I' = IP^{-1} \supsetneq I$. Then by Noetherian induction, $I$ is a product of powers of prime ideals, say $I = \prod P^{a_P}$.

    We get the same for fractional ideals $J = x^{-1} I$.

    Consider the homomorphisms $\{$fractional ideals of $R\}\to\{$fractional ideals of $R_P\}\to\Z$ given by $I \mapsto IR_P$, $(\pi^n)\mapsto n$.

    The composition is $I \mapsto v_P(I)$, and if $Q \neq P$ then $v_P(Q) = 0$.

    So $\{$fractional ideals of $R\}\to \bigoplus_P \Z$ maps $\prod P^{a_P}$ to $(a_P)_P$. Hence the $a_P$ are unique and this is an isomorphism.
    \item By unique factorisation of ideals (part 1.), $\prod P^{a_P} \cap \prod P^{b_P} = \prod P^{\max{a_P,b_P}}$. So if $I+J=R$, then $IJ = I \cap J$, and so by CRT, $R/IJ \cong R/I\cap J \cong R/I \times R/J$, and we are done in this case.

    Hence this step reduces to showing that $(R:P^{n+1}) = (R:P)(R:P^n)$.

    Now localising at $P$ gives $P^n \cong R_P/P^nR_P$, so WLOG $R$ is local, hence a DVR, and $P = (\pi)$.

    Then $R/(\pi^n)\cong (\pi)/(\pi^{n+1})$ via multiplication by $\pi$, and hence $(R:P^{n+1}) = (R:P)(P:P^{n+1}) = (R:P)(R:P^n)$.
  \end{enumerate}
\end{proof}
The quotient group $Cl(R) \coloneqq \{\text{fractional ideals of $R$}\}/\{\text{principal fractional ideals of $R$}\}$ is the \emph{class group} (or \emph{Picard group}) of $R$.

If $K$ is a number field, then we write $Cl(K) = Cl(O_K)$, the \emph{ideal class group} of $K$.
\begin{theorem}
  For $K$ a number field, $Cl(K)$ is finite.
\end{theorem}
The proof will come later.

\subsection{Places of Number Fields}
Recall Ostrowski's theorem, which says that $V_\Q = \{p:p\text{ prime}\}\cup\{\infty\}$.

Let $\p \subset \O_K$ be a nonzero prime ideal. Then $\p$ determines a discrete valuation $v_\p$ of $K$, and so a non-archimedean absolute value $\abs{x}_\p = r^{-v_\p(x)}$ where $r>1$.
\begin{theorem}
  This gives a bijection $\{\text{primes of $O_K$}\} \to V_{K,f}$.
\end{theorem}
\begin{proof}
  Let $P\neq Q$. Then there exists $x \in P \setminus Q$, and then $|x|_P < 1 = |x|_Q$, so $|\cdot|_P, |\cdot|_Q$ are inequivalent, and the map is injective.

  Let $\abs{\cdot}$ be a non-archimedean absolute value on $K$, with valuation ring $R = \{x \in K : |x|\leq 1\}$. As $\abs{\cdot}$ is non-archimedean, $\Z\subset R$ and hence $R \supset O_K$ as $R$ is integrally closed. So $R \supset O_{K,P}$ for some prime $P = \mathfrak{m}_R \cap O_K$. Hence $R = O_{K,P}$ because by \textbf{1.1}, $O_{K,P}$ is a maximal subring of $K$.

  Hence $\abs{\cdot}$ and $\abs{\cdot}_P$ are equivalent.
\end{proof}
For $v \in V_{K,f}$, write $P_v$ for the corresponding prime deal of $O_K$. Then $K_v$, the completion of $K$ at $v$, is a complete discretely valued field, with valuation ring $O_v$ or $O_{K_v} \subset K_v$, not to be confused with $O_{K,P_v}$.

The normalised discrete valuation will be the one for which $v:K^\times \surjection \Z$ is surjective. We will denote by $\pi_v \in O_v$ any generator of the maximal ideal. We often assume $\pi_v \in K$. Then $v(\pi_v) = 1$.

We write $k_v = O_K/P_v \cong O_v/(\pi^v)$, is finite, of order $q_v = p^{f_v}$ for a rational prime $p$ divisible by $v$.

We then normalise the absolute value so that $\abs{x}_{v} = q_v^{-v(x)}$, so that $\abs{\pi_v}_v = \frac{1}{q_v}$.

For infinite places, we have the unique infinite place $\infty$ of $\Q$, with $\Q_\infty = \R$. So then $K \otimes_\Q \R \cong \prod_{v \in V_{K,\infty}} K_v$. Each $K_v$ is a finite extension of $\R$, so is one of $\R$ and $\C$. These correspond to the cases where we say $v$ is real or complex respectively.

In the complex case, since $K \subset K_v$ is dense, we cannot have $K \subset \R$. On the other hand, by Galois theory $\Sigma_K = \{\text{homomorphisms }K \injection \C\}$ has order $n = [K:\Q]$, and $K \otimes_\Q \C \cong \prod_{\sigma \in \Sigma_K} \C$. Complex conjugation acts on both sides by $x \otimes z \mapsto x \otimes \bar{z}$ and $(z_\sigma)_\sigma \mapsto (\overline{z_{\bar{\sigma}}})_\sigma$.

Let $\sigma_1, \ldots, \sigma_{r_1} : K \injection \R$, $\sigma_{r_1+1} = \bar{\sigma}_{r_1+r_2+1}, \ldots, \sigma_{r_1+r_2} = \bar{\sigma_n} : K \injection \C$ where $r_1+2r_2 = n$. Then, taking fixed points under complex conjugation,
\[K \otimes_\Q \R \cong \prod_{\sigma \text{ real}} \R \times \prod_{(\sigma, \bar{\sigma})} \{(z,\bar{z}) \in \C \times \C\} \cong \R^{r_1}\times \C^{r_2}\]
Therefore we have:
\begin{theorem}
  There is a bijection
  \[\Sigma_K / (\sigma \thicksim \bar{\sigma}) \xrightarrow{\sim} V_{K,\infty}\]
  given by $\sigma \mapsto |\sigma(\cdot)|$, where $\abs{\cdot}$ is the Euclidean absolute value in $\R$ or $\C$.
\end{theorem}
We write $K_\infty \coloneqq K \otimes_\Q \R$, which is canonically isomorphic to $\prod_{v \in V_{K, \infty}} K_v$ and noncanonically isomorphic to $\R^{\#\{\text{real }v\}}\times \C^{\#\{\text{complex }v\}}$.

We now choose the normalised absolute values such that, if $v$ is real corresponding to $\sigma:K \injection \R$, then $\abs{x}_v = \abs{\sigma(x)}_\infty$, and if $v$ is complex, then $\abs{x}_v = \sigma(x)\bar{\sigma}(x) = |\sigma(x)|^2$.

If $v$ is finite and $w|v$, then $L_w/K_v$ is a finite extension of non-archimedean local fields, and $[L_w:K_v] = e(w|v)f(w|v)$.

If $v$ is infinite and $w|v$, then $L_w/K_v = \begin{cases} \R/\R & f= e=1\\ \C/\C & f=e=1 \\\C/\R & v\text{ ramified, } e=2,f=1\end{cases}$
\begin{proposition}
  Let $x \in L, v \in V_K$. Then:
  \[|N_{L/K}(x)|_v = \prod_{w|v}|x|_w\]
\end{proposition}
\begin{proof}
  $N_{L/K}(x) = \prod_{w|v}N_{L_w/K_v}(x)$, so it is enough to show that $|N_{L_w/K_v}(x)|_v = |x|_w$.

  For $v$ a finite place, it is enough to take $x = \pi_w \in L$. Then:
  \begin{align*}
    |N_{L_w/K_v}(\pi_w)|_v &= |u\pi_v^{f(w|v)}|_v\\
    &= q_v^{-f(w|v)} = q_w^{-1} = |\pi_w|_w
  \end{align*}

  For $v$ an infinite place, we need only consider $L_w/K_v \cong \C/\R$. But $N_{\C/\R}(z) = z\bar{z}$ by definition.
\end{proof}
\begin{theorem}[Product Formula]
  Let $x \in K^\times$. Then:
  \[|x|_v = 1 \text{ for all but finitely many $v$}\]
  and
  \[\prod_{v\in V_K}|x|_v = 1\]
\end{theorem}
\begin{proof}
  Write $x = a/b$ where $a,b \in O_K^\times$. Then $\{v \in V_K: |x|_v \neq 1\}\subset V_{K, \infty} \cup\{v \in V_{K,f} : v(a)\text{ or }v(b) >0\}$, a finite set.

  Now $\prod_{v\in V_K}|x|_v = \prod_{p\leq \infty} \left(\prod_{v|p}|x|_v) = \prod_{p\leq \infty}|N_{K/\Q}(x)|_p$, so it is enough to prove this for $K= \Q$.

  By multiplicativity, we reduce to $x = q$, a prime, or $x= -1$. In the former case, $|q|_p = \begin{cases} \frac1q & p =q\\1 & p \neq q, \infty\\ q& p=\infty \end{cases}$, and in the latter, $|-1|_p = 1$ for all $p \leq \infty$. Hence the product over all of these is 1.
\end{proof}
\textbf{Remark.} In $\R$, the standard measure $dx$ transforms under $\R^\times$ as $d(ax) = |a|dx$. In $\C$, the standard measure is $dxdy$. This transforms under $\C^{\times}$ as $dxdy \mapsto |a|^2dxdy$. In both these cases, we see the scaling is the normalised AV of $a$.

Fact: on $K_v$ for any $v$ there is a translation-invariant measure, the ``Haar measure'', $d_v(x)$. Then for all $a \in K_v^\times$, $d_v(ax) = |a|_v d_v(x)$, where $\abs{\cdot}_v$ is a normalised AV.

\section{Different and Discriminant}
Suppose $R \subset S$ are commutative rings with 1, such that $S$ is a free $R$-module of finite rank $n \geq 1$. Then we have a trace map $\Tr_{S/R}:S \to R$, the trace of the $R$-linear map $y \mapsto xy$.

If $x_1, \ldots, x_n$ are elements of $S$, define the discriminant:
\[\disc_{S/R}(x_i) = \disc(x_i) = \det \Tr_{S/R} (x_ix_j) \in R\]

If $y_i = \sum_{j=1}^n r_{ji}x_j$, then $\Tr_{S/R}(y_iy_j) = \sum_{k,\ell} r_{ki}r_{\ell j}\Tr_{S/R}(x_kx_\ell)$.

Hence $\disc(y_i) = \det(r_{ij})^2 \disc(x_i)$.

Writing $S = \bigoplus_{i=1}^n Re_i$, we also define $\disc(S/R) = (\disc(e_i)) \subset R$, an ideal of $R$, independent of basis by the previous line.

This has the following basic properties:
\begin{itemize}
  \item $S = S_1 \times S_2$ implies $\disc(S/R) = \disc(S_1/R)\disc(S_2/R)$
  \item If $f:R \to R'$ is a ring homomorphism, then $\disc(S\otimes_R R'/R') = f(\disc(S/R))\subset R'$.
  \item If $R$ is a field, then $\disc(S/R)$ is an ideal, so is $R$ or $\{0\}$, and is $R$ iff the $R$-bilinear form $(x,y)\mapsto \Tr_{S/R}(sy)$ is non-degenerate.
\end{itemize}
If $L/K$ is a finite field extension, then $\disc(L/K) = K \iff$ the trace form is non-degenerate, which holds $\iff$ there is some $x \in L$ with non-zero trace, i.e. iff $L/K$ is separable. More generally:
\begin{theorem}
  Let $k$ be a field and $A$ a finite dimensional $k$-algebra. Then $\disc(A/k) \neq 0$ (so $=k$) if and only if $A = \prod K_i$ where $K_i/k$ are finite separable field extensions.
\end{theorem}
\begin{proof}
  We can write $A = \prod A_i$ where $A_i$ are indecomposable, so local, $k$-algebras, and so we may assume $A$ is local with maximal ideal $\m$.

  If $\m = 0$, i.e. $A$ is field, then this is reduced to the previous statement.

  If not, then every element of $\m$ is nilpotent, so there is $x \in \m\setminus 0$ nilpotent, and so the endomorphism $y \mapsto xy$ of $A$ is nilpotent, and for all $r \in A$, so is $y\mapsto (rx)y$, and hence for all $r \in A$, $\Tr_{A/k}(rx) = 0$, giving a degenerate trace form so a zero discriminant.
\end{proof}
If $R$ is a Dedekind domain, $K = \Frac(R)$, and $L/K$ is finite separable with $S$ the integral closure of $R$ in $L$, then we say $S/R$ is an \emph{extension of Dedekind domains}. Then $S$ is a finitely generated $R$-module, but needn't be free.
\begin{proposition}
  $S$ is a \emph{locally free} $R$-module of rank $n = [L:K]$ (i.e. for all $P \subset R, S_P \cong R_P^n$).
\end{proposition}
\begin{proof}
  $S \subset L$ so $S$ is torsion free, hence so is $S_P$, and $R_P$ is a PID, so $S_P$ is free. $S$ spans $L$ as a $K$-vector space, so $S$ has rank $\dim_K L = n$.
\end{proof}
\begin{lemma}
  If $x \in S$, $\Tr_{L/K}(x) \in R$.
\end{lemma}
\begin{proof}
  If $R$ is local, then $S$ is a free $R$-module, so $\Tr_{L/K}(x) = \Tr_{S\otimes_R K/K}(x\otimes 1) = \Tr_{S/R}(x) \in R$.

  So, in general, for all $0 \neq P \subset R$, $y = \Tr_{L/K}(x) \in R_P$, and $\cap_P R_P = \{x \in K : v_P(x) \geq 0 \forall P\} = R$.
\end{proof}
Then there are 2 equivalent definitions of $\disc(S/R):$

\textbf{Definition.} $\disc(S/R) \coloneqq$ the ideal of $R$ generated by $\{\disc_{L/K}(x_1, \ldots, x_n) : x_1, \ldots, x_n \in S\}$. If $S$ is free, this gives the previous definition.

As $S \otimes_R K = L$ is separable over $K$, $\disc(L/K) \neq 0$, and so $\disc(S/R) \neq \{0\}$.
\begin{proposition}
  $\disc(S/R) R_P = \disc(S_P/R_P)$ for all $P$.
\end{proposition}
\begin{proof}
  We claim there exist $x_1, \ldots, x_n \in S$ which are an $R_P$-basis for $S_P$. Certainly, there is such a basis in $S_P$, say $e_1, \ldots, e_n$. Now let:
  \[\mathscr{Q} \coloneqq \{\text{primes }Q\subset S : v_Q(e_i) < 0\text{ for some $i$}\}\]
  $\mathscr{Q}$ is finite. Then by the CRT, there are $a_i \in S$ such that $v_Q(a_i) + v_Q(e_i) \geq 0$ for all $Q \in \mathscr{Q}$, and $v_P(a_i) \geq 1$.

  Then $x_i = a_ie_i \in S$, and $x_i\equiv e_i$ mod $PS$. So $(x_i)$ is an $R/P$-basis for $S/PS = S_P/PS_P$. So $(x_i)$ is an $R_P$ basis for $S_P$.

  Hence the discriminant $\disc(S_P/R_P) = \disc(x_i)R_P$, and hence $\disc(x_i) \in \disc(S/R)$. So $\disc(S_P/R_P) \subset \disc(S/R)R_P$, and the other inclusion is obvious.
\end{proof}
The alternate definition of $\disc(S/R)$ is:
\begin{itemize}
  \item if $x_1, \ldots, x_n \in S$ is a $K$-basis for $L$, then $\disc_{L/K}(x_i) \neq 0$.

  Let $\mathcal{P} = \{P \subset R : v_P(\disc_{L/K}(x_i)) > 0\}$, a finite set. So for all $P \notin \mathcal{P}$, $\disc(S_P/R_P) = R_P$. Then we can define $\disc(S/R) = \prod_{P \in \mathcal{P}} P^{v_P(\disc(S_P/R_P))}$.
\end{itemize}
This is equivalent to the previous definition by \textbf{3.4}.
\begin{theorem}
  $v_P(\disc(S/R)) = 0$ if and only if $P$ is unramified in $S$ and for all $Q \subset S$ over $P$, the residue field extension $\frac{S}{Q}/\frac{R}{P}$ is separable.
\end{theorem}
\begin{proof}
  We may assume $R$ is local, so that $S$ is a free $R$-module. We know $PS = \prod_Q Q^{e_Q}$. So:
  \[ S \otimes_R (R/P) \cong S/PS \cong \prod_{Q} S/Q^{e_Q}\]
  So $v_P(\disc(S/R)) = 0$ if and only if $\disc(\frac{S}{PS}/\frac{R}{P}) = \frac{R}{P}$ if and only if each $S/Q^{e_Q}$ is a finite separable field extension of $R/P$, if and only if, for all $Q$, $e_Q = 1$ and $\frac{S}{Q}/\frac{R}{P}$ is separable.
\end{proof}
\begin{proposition}
  Let $P \subset R$. Then $v_P(\disc(S/R)) = \sum_{Q\supset P}v_P(\disc(\hat{S}_Q/\hat{R}_P))$.
\end{proposition}
\begin{proof}
  By \textbf{3.4} we may assume $R$ is local, so $S$ is a free $R$-module, and then:
  \[S\otimes_R \hat{R} \cong \prod \hat{S}_Q\]
  So $v_P(\disc(S/R)) = v_P(\disc(S\otimes_R \hat{R}/\hat{R})) = \sum_Q v_P(\disc(\hat{S}_Q/\hat{R}))$.
\end{proof}
\textbf{Definition.} The \emph{inverse different} $\mathscr{D}_{S/R}^{-1}$ of an extension $S/R$ of Dedekind domains is:
\[\mathscr{D}_{S/R}^{-1} = \{x \in L : \forall y \in S, \Tr_{L/K}(xy) \in R\}\]
This is the dual of $S$ with respect to the trace form $(x,y) \mapsto \Tr_{L/K}(xy)$.

This is clearly an $S$-submodule of $L$. If $\bigoplus_{i=1}^n Rx_i \subset S$, let $(y_i)$ be the dual basis to $(x_i)$ for the trace form, i.e. $\Tr_{L/K}(x_iy_j) = \delta_{ij}$. Then $S \subset \mathscr{D}_{S/R}^{-1} \subset \bigoplus_{i=1}^n Ry_i$, so $\mathscr{D}_{S/R}^{-1}$ is a fractional ideal (since it is finitely generated), and its inverse $\mathscr{D}_{S/R}$ is an ideal of $S$, called the \emph{different}.
\begin{proposition}\hspace*{0cm}
  \begin{enumerate}
    \item $P\subset R \implies \mathscr{D}_{S_P/R_P} = \mathscr{D}_{S/R}S_P$
    \item $N_{L/K}(\mathscr{D}_{S/R}) = \disc(S/R)$
    \item $Q \subset S$ lying over $P\subset R$. Then $v_Q(\mathscr{D}_{S/R}) = v_Q(\mathscr{D}_{\hat{S}_Q/\hat{R}_P})$
  \end{enumerate}
\end{proposition}
\begin{proof}\hspace*{0cm}
  \begin{enumerate}
    \item Exercise. Same idea as \textbf{3.4}.
    \item By (1) and \textbf{3.4}, we can suppose $R$ is local. Then $S$ is a PID by \textbf{2.1.iii}, so the inverse different $\mathscr{D}^{-1}_{S/R} = x^{-1} S$ for some $0 \neq x \in S$. Let $(e_i)$ be a basis for $S/R$. Then there exists a basis $(e_i')$ for $S/R$ such that
    \[\Tr_{L/K}(e_i x^{-1} e'_j) = \delta_{ij}\]
    Let $x^{-1}e_j = \sum_{k} b_{kj}e_k$ where $b_{kj} \in K$. Then the ideal:
    \begin{align*}
      (1) &= (\det[\Tr_{L/K}(e_ix^{-1}e_j')]) \\
      &= (\det (\Tr_{L/K}(e_ie_j))\det(b_ij)) \\
      &= \det(b_{ij}) \disc(S/R)
    \end{align*}
    But $N_{L/K}(x^{-1}) = u \det(b_{ij})$ for some unit in $R$. So $(1) = (N_{L/K}(x^{-1}))\disc(S/R)$. Since we are in Dedekind domains we can cancel $\disc(S/R)$, giving the result.
  \end{enumerate}
\end{proof}
\end{document}
