\documentclass[10pt,a4paper]{article}
\author{Harry Armitage}

%\usepackage[utf8]{inputenc}
\usepackage{amsmath}
\usepackage{amsfonts}
\usepackage{amssymb}
\usepackage{amsthm}
\usepackage{float}
\usepackage{mathtools}
\usepackage{geometry}[margin=1in]
\usepackage{xspace}
\usepackage{tikz}
\usepackage{mathrsfs}
\usetikzlibrary{shapes, arrows, decorations.pathmorphing, ducks, automata}
\usepackage[parfill]{parskip}
\usepackage{subcaption}
\usepackage{stmaryrd}
\usepackage{marvosym}
\usepackage{dsfont}
\usepackage{pgfplots}
\usepackage{enumitem}
\usepackage{calc}
\usepackage{tikz-cd}
\usepackage{hyperref}
\usepackage[usestackEOL]{stackengine}

\usepackage{fontspec}
\usepackage{newpxtext, newpxmath}
\usepackage{anyfontsize}

\hypersetup{
    colorlinks,
    citecolor=black,
    filecolor=black,
    linkcolor=black,
    urlcolor=black
}

\newcommand{\f}[1]{\mathfrak{#1}}
\newcommand{\p}{\f{p}}

\newcommand{\st}{\text{ s.t. }}
\newcommand{\contr}{\lightning}
\newcommand{\im}{\mathfrak{i}}
\newcommand{\R}{\mathbb{R}}
\newcommand{\Q}{\mathbb{Q}}
\renewcommand{\C}{\mathbb{C}}
\newcommand{\F}{\mathbb{F}}
\newcommand{\K}{\mathbb{K}}
\newcommand{\N}{\mathbb{N}}
\newcommand{\Z}{\mathbb{Z}}
\renewcommand{\P}{\mathbb{P}}
\renewcommand{\H}{\mathds{H}}
\renewcommand{\O}{\mathcal{O}}
\newcommand{\A}{\mathbb{A}}
\newcommand{\D}{\mathbb{D}}
\renewcommand{\G}{\mathbb{G}}
%\newcommand{\nequiv}{\not\equiv}
\newcommand{\powset}{\mathcal{P}}
\renewcommand{\th}[1][th]{\textsuperscript{#1}\xspace}
\newcommand{\from}{\leftarrow}
\newcommand{\legendre}[2]{\left(\frac{#1}{#2}\right)}
\newcommand{\ow}{\text{otherwise}}
\newcommand{\imp}[2]{\underline{\textit{#1.}$\implies$\textit{#2.}}}
\let\oldexists\exists
\let\oldforall\forall
\renewcommand{\exists}{\oldexists\;}
\renewcommand{\forall}{\;\oldforall}
\renewcommand{\hat}{\widehat}
\renewcommand{\tilde}{\widetilde}
\newcommand{\one}{\mathds{1}}
\newcommand{\under}{\backslash}
\newcommand{\injection}{\hookrightarrow}
\newcommand{\surjection}{\twoheadrightarrow}
\newcommand{\isomarrow}{\mathrel{\setstackgap{S}{-0.5pt}\ensurestackMath{\Shortstack{\scriptstyle\sim\\ \longrightarrow}}}}
\newcommand{\jacobi}{\legendre}
\newcommand{\floor}[1]{\lfloor #1 \rfloor}
\newcommand{\ceil}[1]{\lceil #1 \rceil}
\newcommand{\cbrt}[1]{\sqrt[3]{#1}}
\renewcommand{\angle}[1]{\langle #1 \rangle}
\newcommand{\dbangle}[1]{\angle{\angle{#1}}}
\newcommand{\wrt}{\text{ w.r.t. }}
\newcommand{\abs}[1]{\lvert#1\rvert}
\newcommand{\norm}[1]{\lVert#1\rVert}
\newcommand*\circled[1]{\tikz[baseline=(char.base)]{
      \node[shape=circle,draw,inner sep=2pt] (char) {#1};}
}
\renewcommand{\epsilon}{\varepsilon}
\newcommand{\trianglerightneq}{\mathrel{\ooalign{\raisebox{-0.5ex}{\reflectbox{\rotatebox{90}{$\nshortmid$}}}\cr$\triangleright$\cr}\mkern-3mu}}
\newcommand{\triangleleftneq}{\mathrel{\reflectbox{$\trianglerightneq$}}}

\DeclareMathOperator{\ex}{ex}
\DeclareMathOperator{\id}{id}
\DeclareMathOperator{\upper}{Upper}
\DeclareMathOperator{\dom}{dom}
\DeclareMathOperator{\disc}{disc}
\DeclareMathOperator{\charr}{char}
\DeclareMathOperator{\Image}{im}
\DeclareMathOperator{\ord}{ord}
\DeclareMathOperator{\lcm}{lcm}
\DeclareMathOperator{\aut}{Aut}
\DeclareMathOperator{\diag}{diag}
\DeclareMathOperator{\stab}{stab}
\DeclareMathOperator{\trace}{trace}
\DeclareMathOperator{\ecl}{ecl}
\DeclareMathOperator{\Span}{Span}
\DeclareMathOperator{\Gal}{Gal}
\DeclareMathOperator{\Aut}{Aut}
\DeclareMathOperator{\Frob}{Frob}
\DeclareMathOperator{\Det}{Det}
\let\div\relax
\DeclareMathOperator{\div}{div}
\DeclareMathOperator{\Div}{Div}
\let\Re\relax
\let\Im\relax
\DeclareMathOperator{\Re}{\mathfrak{Re}}
\DeclareMathOperator{\Im}{\mathfrak{Im}}
\DeclareMathOperator{\Frac}{Frac}
\DeclareMathOperator{\Pic}{Pic}
\DeclareMathOperator{\ann}{ann}
\DeclareMathOperator{\Ass}{Ass}
\DeclareMathOperator{\intt}{int}
\DeclareMathOperator{\Hom}{Hom}
\DeclareMathOperator{\End}{End}
\DeclareMathOperator{\tr}{tr}
\DeclareMathOperator{\Tr}{Tr}
\DeclareMathOperator{\Spec}{Spec}
\DeclareMathOperator{\height}{ht}
\DeclareMathOperator{\rank}{rank}
\DeclareMathOperator{\Art}{Art}
\DeclareMathOperator{\gr}{gr}
\DeclareMathOperator{\Tor}{Tor}
\DeclareMathOperator{\Ext}{Ext}
\DeclareMathOperator{\coker}{coker}

\let\emph\relax
\DeclareTextFontCommand{\emph}{\bfseries\em}

\newtheorem{theorem}{Theorem}[section]
\newtheorem{lemma}[theorem]{Lemma}
\newtheorem{corollary}[theorem]{Corollary}
\newtheorem{proposition}[theorem]{Proposition}
\newtheorem{conjecture}[theorem]{Conjecture}
\newtheorem{definition}[theorem]{Definition}

\definecolor{burgundy}{rgb}{0.5, 0.0, 0.13}

\tikzset{sketch/.style={decorate,
 decoration={random steps, amplitude=1pt, segment length=5pt},
 line join=round, draw=black!80, very thick, fill=#1
}}


\title{Ramsey Theory}
\begin{document}
\maketitle
\tableofcontents
\newpage
\section{Monochromatic Systems}
Here, we let $\N = \{1,2,3,\ldots\}$, and write $[n] = \{1,2,\ldots, n\}$. For a set $X$ and $r \in \N$, we write $X^{(r)} = \{A \subset X : |A|=r\}$, the collection of all $r$-sets in $X$.

Suppose we are given a $2$-colouring of $\N^{(2)}$, i.e. a function $C: \N^{(2)} \to \{1,2\}$. We can think of this being a colouring of the edges of the complete graph on $\N$. Can we find an infinite monochromatic $M$, i.e. a set $M \subset \N$ such that $C$ is constant on $M^{(2)}$.

\textbf{Examples}
\begin{enumerate}
  \item Colour $ij$ (shorthand for the set ${i,j}$) red if $i+j$ is even, and blue if $i+j$ is odd. Here, the answer is yes - take $M = 2\N = \{2,4,6,\ldots\}$.
  \item Colour $ij$ red if $\max\{n: 2^n|i+j\}$ is even, and blue if it is odd. The answer is yes - $M = \{4^0,4^1,4^2,4^3,\ldots\}$.
  \item Colour $ij$ red if $i+j$ has an even number of distinct prime factors, and blue if odd. This is more difficult. To save some time, we shall use the following theorem to answer every question of this form:
\end{enumerate}
\begin{theorem}[Ramsey's Theorem]
  Let $C: \N^{(2)} \to \{1,2\}$ be a 2-colouring of $\N^{(2)}$. Then there exists an infinite monochromatic subset of $\N$.
\end{theorem}
\begin{proof}
  Pick $a_1 \in \N$. Then there are infinitely many edges out of $a_1$, so infinitely many have the same colour - say all edges from $a_1$ to the infinite set $B_1$ have colour $c_1$.

  Now pick $a_2 \in B_1$. There must be some infinite set $B_2 \subseteq B_1\setminus\{a_2\}$ with all edges $a_2$ to $B_2$ are the same colour, say $c_2$, and repeat inductively.

  We then obtain $a_1, a_2, a_3, \ldots$ and colours $c_1, c_2, c_3, \ldots$ such that $a_ia_j$ for $i<j$ has colour $c_i$. Now infinitely many of the $c_i$ must be the same colour, say $c$. Then we may take $M = \{a_i: c_i = c\}$.
\end{proof}

\textbf{Remarks}
\begin{enumerate}
  \item This is sometimes called a 2-pass proof - we went through all the numbers to build the sequence $a_1, a_2, \ldots$.
  \item In example 3, no explicit example is known.
  \item The exact same proof works for $n$ colours. Alternatively, we could deduce this from Ramsey's theorem + induction - view the colours as `1' and `2 or 3 or ...'. If the infinite set is coloured 1, we are done, otherwise repeat with the $n-1$ colours remaining.
  \item An infinite monochromatic set is more than having arbitrarily large finite monochromatic sets. For example, make $\{1\}, \{2,3\}, \{4,5,6\}, \ldots$ all monochromatic blue sets, but make all edges between them red. Then there is no infinite monochromatic blue set (there is however an infinite red set - $\{1,3,6,\ldots\}$).
\end{enumerate}

\textbf{Example.}
Any sequence $x_1, x_2, \ldots$ in $\R$ (or any totally ordered set) has a monotone subsequence. This was seen in Analysis I, where the proof worked by fiddling around with lim sups and lim infs. Instead, just colour $ij$ `up' if $x_i \leq x_j$, and `down' if $x_i>x_j$. Then apply Ramsey's theorem.

What if we coloured $\N^{(r)}$, say for $r = 3,4,\ldots$? Given a 2-colouring $C:\N^{(r)}\to \{1,2\}$, must there be an infinite monochromatic set?

For example, take $r = 3$. Colour $ijk$ red if $i|j+k$, and blue if not. Yes - take powers of two. As in the case $r = 2$, we can always do this:
\begin{theorem}[Ramsey for $r$-sets]
  Let $r \in \N$. Then if $C:\N^{(r)} \to \{1,2\}$ is a 2-colouring of $\N^{(r)}$, there exists a monochromatic subset of $\N$.
\end{theorem}
\begin{proof}
  $r = 1$ is trivial - just infinite pigeonhole. $r = 2$ is \textbf{Theorem 1.1}. We then induct on $r$ - suppose our result holds for $r - 1$.

  Pick $a_1 \in \N$. We have an induced colouring $D:(\N\setminus\{a_1\})^{(r-1)} \to \{1,2\}$ given by $D(F) = C(F\cup\{a_1\})$. By induction, there is an infinite $B_1 \subset \N\setminus \{a_1\}$, monochromatic for the colouring $D$. That is, $C(F\cup\{a_1\}) = c_1$ for all $F \subset B_1^{(r-1)}$.

  Repeating for $B_1$, we have $a_2 \in B_1$ and an infinite $B_2 \subset B_1 \setminus \{a_2\}$ with all $F\cup\{a_2\}, F \in B_2^{(r-1)}$ have the same colour, $c_2$ say.

  We thus obtain distinct $a_1, a_2, \ldots$ and colours $c_1, c_2, \ldots$ such that $C(a_{i_1}a_{i_2}\ldots a_{i_r}) = c_{i_1}$. But infinitely many of the $c_i$ agree, say all $i \in I$, and so $\{a_i:i \in I\}$ is our set.
\end{proof}
\textbf{Example.} We saw from \textbf{Theorem 1.1} that, given points $(1, x_1), (2,x_2), \ldots$ in $\R^2$, there is a subsequence such that the induced function is monotone. In fact, we can ensure that the induced function is either convex or concave. Just 2-colour the triples $ijk$ in $\N^{(r)}$ red if the points $(i,x_i), (j, x_j), (k, x_k)$ are convex, and blue if they are concave (pick either if they are collinear). Then apply \textbf{1.2}.

Surprisingly, the infinite version of Ramsey implies the finite version.
\begin{theorem}[Finite Ramsey]
  For all $m, r \in \N$, there is some $n \in \N$ such that whenever $[n]^{(r)}$ is 2-coloured, there is a monochromatic $m$-set.
\end{theorem}
\begin{proof}
  Suppose not. We'll show that there is a 2-colouring of $\N^{(r)}$ without a monochromatic $m$-set, contradicting \textbf{Theorem 1.2}. For each $n \geq r$, we have a 2-colouring $C_n$ or $[n]^{(r)}$ with no monochromatic $m$-set. We would like to take their union to get a bad colouring of $\N^{(r)}$, but can't do this, since they might not agree with each other. There are only finitely many ways to 2-colour $[r]^{(r)}$ (two in fact), so infinitely many of the $C_n$ agree on $[r]^{(r)}$ - say $C_n|_{[r]^{(r)}} = D_r$ for all $n \in A_1$ and some colouring $D_r: [r]^{(r)} \to \{1,2\}$, where $A_1$ is infinite.

  There are only finitely many ways to 2-colour $[r+1]^{(r)}$, so infinitely many of the $C_n, n \in A_1$ agree on $[r+1]^{(r)}$ - say they are $D_{r+1}$ on $[r+1]^{(r)}$ for all $n \in A_2 \subset A_1$. Continue inductively.

  We obtain colourings $D_n : [n]^{(r)}\to \{1,2\}$ for each $n \geq r$, such that:
  \begin{enumerate}
    \item $D_n$ has no monochromatic $m$-set, as $D_n = C_{n'}|_{[n]^{(r)}}$ for some $n' \geq n$, where $C_{n'}$ has no monochromatic $m$-set.
    \item The $D_n$ are nested in the following way: $D_{n'}|_{[n]^{(r)}} = D_n$ for all $n' > n$.
  \end{enumerate}
  Now put $C:\N^{(r)} \to \{1, 2\}; C(F) = D_n(F)$ for any $n \geq \max F$. This is well defined by property 2 and has no monochromatic $m$-set by 1. $\contr$
\end{proof}

\textbf{Remark.} This is called a \emph{compactness argument}: what we are showing is that the space $\{1,2\}^{\N}$ of 2-colourings, with the product topology, is (sequentially) compact. Note also that this proof gives no information on how large $n = n(m,r)$ has to be. There are direct proofs giving bounds.

What if we colour $\N^{(2)}$ with infinitely many colours, i.e. we have $C: \N^{(2)} \to X$, some set $X$. Of course we do not get an infinite monochromatic set ever time - e.g. colour every pair with a different colour. However, this colouring is injective on this set. We might ask, in our quest for order amongst chaos, ``Do we always get an infinite set $m$ son which $C$ is either constant or injective?''

It turns out the answer to this question is also no: let $C(ij) = i$ ($i < j$, as usual). However, this is essentially the only way things can go wrong:
\begin{theorem}[Canonical Ramsey Theorem]
  Let $C$ be a colouring of $\N^{(2)}$ with an arbitrary set of colours. Then there exists an infinite set $M \subset \N$ such that one of:
  \begin{enumerate}
    \item $C$ is constant on $M^{(2)}$.
    \item $C$ is injective on $M^{(2)}$.
    \item For all $ij, k\ell \in M^{(2)}$, $C(ij) = C(k\ell) \iff i=k$.
    \item For all $ij, k\ell \in M^{(2)}$, $C(ij) = C(k\ell) \iff j=\ell$.
  \end{enumerate}
\end{theorem}
Note that this implies Ramsey's theorem, as cases 2,3,4 all require infinitely many colours.
\begin{proof}
  Define a 2-colouring of $\N^{(4)}$ by giving $ijk\ell$ the colour \textit{same} if $C(ij) = c(k\ell)$ and \textit{diff} if $C(ij) = C(k\ell)$.

  By Ramsey for 4-sets, there is an infinite monochromatic $M_1$ for this colouring. If it has colour \textit{same}, then we have case $1$. Indeed, given $ij, k\ell \in M_1^{(2)}$, choose $m<n$ with $m>i,j,k,\ell$. Then $C(ij) = C(mn) = C(k\ell)$.

  Otherwise, $M_1$ has colour \textit{diff}. Now 2-colour $M_1^{(4)}$ by giving $ijk\ell$ colour \textit{same} if $C(jk) = C(i\ell)$, and \textit{diff} otherwise. Again, by Ramsey for 4-sets we have an infinite $M_2\subseteq M_1$ monochromatic for this colouring. Note that it cannot be \textit{same}, as then we would have $i<j<k<\ell<m<n$ with $C(jk) = C(in) = C(\ell m)$, giving a contradiction with the previous colouring.

  Now 2-colour $M_2^{(4)}$, giving $ijk\ell$ \textit{same} if $C(ik) = C(jl)$, and \textit{diff} if not. Again, get monochromatic infinite $M_3$. Now $M_3$ cannot have colour \textit{same} as above, hence it has colour \textit{diff}.

  Now 2-colour $M_3^{(3)}$ by giving $ijk$ colour \textit{same} if $C(ij)=C(jk)$, and \textit{diff} if not. We then have an infinite monochromatic $M_4 \subseteq M_3$. Cannot have $M_4$ of colour \textit{same}, else have $i<j<k<\ell$ with $C(ij) = C(jk) = C(k\ell)\contr$, so $M_4$ has colour \textit{diff}.

  Now 2-colour $M_4^{(3)}$ by giving $ijk$ colour \textit{left-same} if $C(ij) = C(ik)$, and \textit{left-diff} if not. We obtain infinite monochromatic $M_5 \subseteq M_4$.

  Finally, 2-colour $M_5^{(3)}$, giving $ijk$ colour \textit{right-same} if $C(ij) = C(jk)$ and \textit{right-diff} if not. We obtain infinite monochromatic $M_6\subseteq M_5$.

  We then split into the cases:
  \begin{itemize}
    \item \underline{$M_6$ is \textit{left-diff} and \textit{right-diff}:} This is case 2. All edges are different.
    \item \underline{$M_6$ is \textit{left-same} and \textit{right-diff}:} This is case 3. The only edges that agree are ones that share a left vertex.
    \item \underline{$M_6$ is \textit{left-diff} and \textit{right-same}:} This is case 4. The only edges that agree are ones that share a right vertex.
    \item \underline{$M_6$ is \textit{left-same} and \textit{right-same}:} This cannot occur. If we have $i<j<k$ in $M_6$, then $C(ij) = C(ik) = C(jk)$, giving a contradiction.
  \end{itemize}
\end{proof}
\textbf{Remarks.}
\begin{enumerate}
  \item We could use just one colouring, colouring 4-sets $ijk\ell$ with the partition of $[4]^{(2)}$ induced by $C$ on $\{i,j,k,\ell\}$. This would use as many colours as there are partitions of a set of size $\binom{4}{2}$.
  \item Similarly, if $C:\N^{(r)} \to X$ is an arbitrary colouring, we get an infinite monochromatic $M \subset \N$ and a set $I \subset [r]$ such that, for all $x_1, \ldots, x_r, y_1,\ldots, y_r \in M^{(r)}$, we have $C(x_1\ldots x_r) = C(y_1\ldots y_r) \iff x_i = y_i \forall i \in I$. These $2^r$ colourings are called the \emph{canonical colourings} of the $r$-sets. In the case $r = 2$, we have the correspondence:
  \begin{center}
    \begin{tabular}{c|c}
      Case in \textbf{1.4} & $I \subset [2]$\\\hline
      1 & $\emptyset$\\
      2 & $\{1,2\}$\\
      3 & $\{1\}$\\
      4 & $\{2\}$
    \end{tabular}
  \end{center}
\end{enumerate}
\section{Van der Waerden's Theorem}
If we 2-colour $\N$, can we find 3 consecutive points of the same colour? Answer: of course not - just alternate colours! Ok - what about 3 equally spaced points, i.e. an arithmetic progression of length 3? This is not obviously false, but also not obviously true. What about even hoping for monochromatic arithmetic progressions of length $m$ for any $m \in \N$. (Here, we use ``length'' to mean the number of terms, not the number of jumps). This is van der Waerden's theorem, which will be the aim of this section.

By our usual compactness argument, this is the same as: for all $m$, there is $n$ such that whenever $[n]$ is 2-colored, there is a monochromatic AP of length $m$. Indeed, if this is false, then for every $n \geq m$, there is a colouring $C_n:[n]\to \{1,2\}$ with no monochromatic AP of length $m$. Then infinitely many agree on $[m]$, and of those infinitely many agree on $[m+1]$ and so on. Put together those nested restrictions to obtain a 2-colouring of $\N$ with no monochromatic AP of length $m \contr$.

In proving this, one key idea is to generalise: we show that for all $m, k$ there is some $n$ such that whenever $[n]$ is $k$-coloured there is a monochromatic AP of length $m$.

Note: proving a stronger result might be easier, e.g. in a proof by induction

Another key idea: given APs $A_1, \ldots, A_r$, each of length $m$, so $A_i = \{a_i, a_i+d_i, \ldots, a_i+(m-1)d_i\}$, we say they are \emph{focused} at $f$ if $a_i+md_i = f$ for all $i$. For instance, $\{1,4\}$ and $\{5,6\}$ are focused at 7.

If in addition each $A_i$ is monochromatic for a given colouring with no two the same colour, we call them \emph{colour-focused}.

So in a $k$-colouring, if we have APs $A_1, \ldots, A_k$ each of length $m-1$ that are colour focused, then we actually have a monochromatic AP of length $m$, by asking ``What colour is the focus?'' For instance, if we are 2-colouring and the sequences $\{1,4\}, \{5,6\}$ are colour-focused (say $\{1,4\}$ is red and $\{5,6\}$ is blue), then if 7 is red we have $\{1,4,7\}$ and if 7 is blue then we have $\{5,6,7\}$.

We will write $W(m,k)$ for the least $n$ (if it exists) such that whenever $[n]$ is $k$-coloured, there is a monochromatic AP of length $m$.
\begin{proposition}
  For all $k$ there is $n$ such that whenever $[n]$ is $k$-coloured there is a monochromatic AP of length 3.
\end{proposition}
Note - this is contained in van der Waerden's theorem - we prove it here for clarity.
\begin{proof}
  Claim: for all $r \leq k$ there exists $n$ such that whenever $[n]$ is $k$-coloured, we have either
  \begin{itemize}
    \item a monochromatic AP of length 3.
    \item or $r$ colour-focused APs of length 2.
  \end{itemize}
  Then we will be done - put $r = k$ and look at the focus.

  For $r = 1$, this is trivial - take $n=k+1$. Then we have 2 points of the same colour, so they form a monochromatic AP of length 1.

  We now proceed by induction: given $n$ suitable for $r-1$, we'll show that $(k^{2n}+1)2n$ is suitable for $r$.

  So now suppose we are given a $k$-colouring of $[(k^{2n}+1)2n]$ with no monochromatic AP of length 3. Break up this large set into blocks of length $2n$ - call them $B_1, B_2, \ldots, B_{k^{2n}+1}$, where $B_i = [(i-1)\cdot 2n+1, i\cdot 2n]$.

  Inside any interval of length $2n$ we have $r-1$ colour-focused APs of length $2$ (given by the choice of $n$) together with their focus (since we took $2n$).

  There are $k^{2n}$ ways of $k$-colouring a block, and therefore some two blocks are identically coloured - say $B_s$ and $B_{s+t}}$.

  In $B_s$ we have $r-1$ colour focused APs of length 2 - say $\{a_1, a_1+d_!\}, \ldots, \{a_{r-1}, a_{r-1}+d_{r-1}$ focused at $f$. But now the APs $\{a_1, a_1+d_1+2nf\}, \ldots, \{a_{r-1}, a_{r-1}+d_{r-1}+2nf\}$ are colour-focused at $f+4nt$, and $\{f, f+2nt\}$ is also focused there. Hence we have $r$ colour-focused APs of length 2.
\end{proof}
\textbf{Remarks.}
\begin{enumerate}
  \item The idea of looking at the number of ways to colour a block is called a \emph{product argument}.
  \item The proof shows $W(3, k) \leq k^{k^{\iddots^{k^{4k}}}}$ where the tower has length $k$. This is called a \emph{tower-type bound}.
\end{enumerate}
\begin{theorem}[van der Waerden's Theorem]
  For all $m, k$ there exists $n$ such that whenever $[n]$ is $k$-coloured there exists a monochromatic AP of length $m$.
\end{theorem}
\begin{proof}
  Induction on $m$. $m=1$ is trivial - an AP of length $1$ is trivial. $m=2$ is just pigeonhole. $m=3$ was done above.

  We may assume that $W(m-1, k)$ exists for every $k$.

  Claim: for every $r \leq k$, there is an $n$ such that, whenever $[n]$ is $k$-coloured, we have either:
  \begin{itemize}
    \item a monochromatic AP of length $m$.
    \item or $r$ colour-focused APs of length $m-1$.
  \end{itemize}
  Then, as before, we will be done.

  We prove this claim by induction on $r$. $r=1$ is done by the induction hypothesis on $m$ -  we can take $n= W(m-1, k)$.

  Now given $n$ suitable for $r-1$, we'll show that $W(m-1, k^{2n})\cdot 2n$ is suitable for $r$.

  So, given a $k$-colouring of $[W(m-1, k^{2n})2n]$ with no monochromatic AP of length $m$: break up $[W(m-1, k^{2n})2n]$ into blocks of length $2n$, say $B_1, B_2, \ldots, B_{W(m-1, k^{2n})}$, where $B_i = [(i-1)2n + 1, i2n]$.

  Now, the number of ways to $k$-colour a block is $k^{2n}$. So, since we have $W(m-1, k^{n2})$ blocks, we must have $m-1$ equally spaced blocks, all coloured identically - say $B_s, B_{s+t}, \ldots, B_{s+(m-2)t}$.

  Inside $B_s$ we have $r-1$ colour-focused APs of length $m-1$ (by the definition of $n$), together with their focus. Say, $A_1, \ldots, A_{r-1}$ focused at $f$, where $A_i$ has first term $a_i$ and common difference $d_i$. Then the APs $A_1', \ldots, A_{r-1}'$, where $A_i'$ has first term $a_i$ and common difference $d_i + 2nt$, are colour focused at $f+(m-1)2n$. Also, $f, f +2nt, f+2\cdot 2nt, \ldots, f+(m-2)2nt$ is monochromatic of a different colour, focused at $f+(m-1)2n$.

  So we have the $r$ colour focused APs of length $m-1$.
\end{proof}
\subsection{The Ackermann or Grzegorczyk Hierarchy}
This is a sequence of functions $f_1, f_2, \ldots$, each $\N \to \N$, given by:
\begin{itemize}
  \item $f_1(x) = 2x$
  \item $f_{n+1}(x) = f_n^{(x)}(1)$
\end{itemize}
So $f_2(x) = 2^x$. $f_3(x) = 2^{2^{\iddots^2}}$, with the tower having $x$ 2s. We don't have notation for $f_4$, but we can compute $f_4(1) = 2, f_4(2) = 2^2 = 4, f_4(3) = 2^{2^{2^2}} = 65536$, and $f_4(4)$ is a tower of 2s of height 65536.

We say $f:\N \to \N$ if there are $c, d >0$ with $f_n(cx) \leq f(x) \leq f_n(dx)$. So our upper bound on $W(3, k)$ was a function of $k$ of type 3., and in general, our bound on $W(m,k)$ is of type $m$, and our bound on $W(m) = W(m,2)$ as a function of $m$ grows faster than every $f_n$. This is often a feature of such `double inductions', and for a long time it was thought that $W(m)$ does actually grow this fast.

Shelah (1987) found a proof of van der Waerden using induction only on $m$, and his proof gives $W(m,k) \leq f_4(m+k)$. Graham offered \$1000 for a bound on $W(m)$ of type 3. Gowers (1998) showed $W(m) \leq 2^{2^{2^{2^{2^{m+9}}}}}$.

For lower bounds, we know that $W(m) \geq \frac{2^m}{8m}$.
\begin{corollary}
  Whenever $\N$ is finitely coloured, some colour class contains arbitrarily long arithmetic progressions.
\end{corollary}
\begin{proof}
  For each $n$, there is a monochromatic AP of length $n$, of colour $c_n$. Some colour must occur infinitely often.
\end{proof}
\textbf{Remark.} We cannot guarantee an infinite AP - take 1 red, 2 blues, 3 reds, 4 blues, etc. Alternatively, there are only countably many infinite APs. List them in order, e.g. $A_1, A_2, \ldots$. Pick $x_1, y_1 \in A_1$ and colour them differently. Now pick distinct $x_2, y_2 \in A_2$ new, and make colour them differently. However, we do have the following:
\begin{theorem}[Strengthened van der Waerden]
  For every $m$, whenever $\N$ is finitely coloured, there exists an AP of length $m$ that, together with its common difference (i.e. $\{a, a+d, a+2d, \ldots, a+(m-1)d\}\cup\{d\}$), is monochromatic.
\end{theorem}
\begin{proof}
  Induction on $k$, the number of colours. $k=1$ is done.

  Given $n$ suitable for $k-1$, we'll show $W(n(m-1)+1, k)$ is suitable for $k$. So, given a $k$-colouring of $[W(n(m-1)+1, k)]$, we have a monochromatic AP of length $n(m-1)+1$, say $[a, a+d, a+2d, \ldots, a+n(m-1)d]$ red.

  If $d$ is red, then we are done, as $\{a, a+d, \ldots, a+(m-1)d\}\cup\{d\}$ is red.

  But also, if $2d$ is red, then we are done, as $\{a, a+2d, \ldots, a+2(m-1)d\}\cup\{2d\}$ is red. In fact, if any $rd$ for $1 \leq r\leq n$ is red, then we are done in the same way.

  Hence $\{d, 2d, \ldots, nd\}$ is $(k-1)$-coloured. But, by the inductive hypothesis, we are done, as this contains the required sequence by definition of $n$.
\end{proof}
\textbf{Remarks.} From now on we don't care about bounds. The bounds in the above proof are far too big to be reasonably describable on paper. The case $m=2$ is called \emph{Schur's Theorem}: whenever $\N$ is finitely coloured, there are monochromatic $x, y, z$ with $x+y=z$. We can also prove this directly from Ramsey. Indeed, given a $k$-colouring of $\N$, we construct a colouring of $\N^{(2)}$ via $d(ij) = c(j-i)$. Then there is a monochromatic triangle $ijk$. So $c(k-i) = c(k-j) = c(j-i)$, and $(k-j) + (j-i) = k-i$.
\section{The Hales-Jewett Theorem}
Let $X$ be a finite set. A subset $L$ of $X^n$ is called a \emph{line} or \emph{combinatorial line} if there is a non-empty $I \subset [n]$ and $a_i \in X$ for each $i \in [n]\setminus I$, such that $L = \{(x_1, \ldots, x_n) \in X^n: x_i = a_i \forall i \notin I, x_i=x_j \forall i,j \in I\}$. $I$ is called the active coordinates.

For example, in $[3]^2$, the lines are 3 vertical, 3 horizontal, and 1 diagonal. We don't allow the other diagonal, since there the $x$-coordinate goes up and the $y$-coordinate goes down. There is no assumed ordering on $X$, and so we want lines to be invariant under reordering. This other diagonal would cease to be a line if we reordered $X$, and hence we exclude it.
\begin{theorem}[Hales-Jewett Theorem]
  For all $m, k$ there exists $n$ such that, whenever $[m]^n$ is $k$-coloured, there is a monochromatic line.
\end{theorem}
\textbf{Remarks.}
\begin{enumerate}
  \item The least such $n$ is denoted $HJ(m,k)$.
  \item This essentially says that $m$-in-a-row noughts-and=crosses, played in enough dimensions, cannot end in a draw. As an exercise, show it is a win for the first player.
  \item Hales-Jewett (HJ) implies VDW. Just map $X^n$ linearly into $\N$, e.g. add up the coordinates. Indeed, given a $k$ colouring of $\N$ induce a $k$-colouring of $[m]^n$ by $d((x_1, \ldots, x_m)) = c(x_1+\ldots+x_m)$. So there is a monochromatic line $L$ in $[m]^n$, which corresponds to a monochromatic AP of length $m$ in $\N$, with common difference given by the number of active coordinates of $L$.
\end{enumerate}
For a line $L$ in $[m]^n$ write $L^-$ for its first point and $L^+$ for its last point. We say $L_1, \ldots, L_r$ are \emph{focused} at $f$ if $L_i^+ = f$ for all $i$, and \emph{colour focused} if they are each monochromatic (except at $f$) of different colours.
\begin{proof}[Proof of Hales-Jewett]
  We will do this by induction on $m$. $m=1$ is trivial.

  Now, given $m > 1$, we may assume $HJ(m-1, k)$ exists for all $k$.

  Claim: for all $r \leq k$ there exists $n$ such that, whenever $[m]^n$ is $k$-coloured there is either:
  \begin{itemize}
    \item a monochromatic line
    \item or $r$ colour-focused lines
  \end{itemize}
  If we can prove this then we're done, since then we put $r=k$ and look at the focus.

  We then prove this claim by induction on $r$. If $r=1$, then done, by taking $n = HJ(m-1,k)$.

  Given $n$ suitable for $r-1$, we'll show $n + HJ(m-1, k^{m^n})$ is suitable for $r$. Write $n'$ for $HJ(m-1, k^{m^n})$. So, given a $k$-colouring of $[m]^{n+n'}$ with no monochromatic line, view $[m]^{n+n'}$ as $[m]^n \times [m]^{n'}$. At each point of $[m]^{n'}$ there are $k^{m^n}$ ways to colour $[m]^n$.

  So, by choice of $n'$, we have a line $L$ in $[m]^{n'}$ such that, for all $a \in [m]^n$ and $b,b' \in L\setminus\{L^+\}$ we have $c(a,b) = c(a,b') = c'(a)$, say.

  By definition of $n$, $c'$ has $r-1$ colour focused lines, say $L_1, \ldots, L_{r-1}$, with active coordinate sets $J_1, \ldots, J_{r-1}$, focused at $f$. Let $L$ have active coord set $I$. Then the lines $L_1', \ldots, L_{r-1}^'$ where $L_i'$ starts at $L_i', L^-$ and has active coord set $J_i \cup I$ are colour focused at $(f, L^+)$, and also the line starting at $(f, L^-)$ with active coords set $I$, are a set of $r$ colour focused lines at $(f, L^+)$. Hence we are done.
\end{proof}
This is essentially the van der Waerden proof, just slightly rephrased.

A \emph{d-dimensional subspace} of $X^n$ is a set $S \subset X^n$ such that there exist disjoint $I_1, \ldots, I_d \subseteq [n]$, and points $a_i \in X$ for each $i \in [n]\setminus(I_1\cup\ldots\cup I_d)$, such that every point in $S$ has $i\th$ coordinate $a_i$ if $i \notin I_1\cup\ldots\cup I_d$, and if $i,j \in I_k$ for some $k$ then $x_i=x_j$.
\begin{theorem}[Extended Hales-Jewett Theorem]
  For any $m,k,d$, there is an $n$ such that any $k$-colouring of $[m]^n$ contains a monochromatic $d$-dimensional subspace.
\end{theorem}
\begin{proof}
  View $X^{dn}$ as $(X^d)^n$, i.e. $n$-coordinates, each of which is a point in $X^d$.

  Clearly any line in $(X^d)^n$ corresponds to a $d$-dimensional subspace of $X^{dn}$, so we are done -  we can take $n=dHJ(m^d,k)$.a
\end{proof}
Let $S$ be a finite subset of $\N^d$. A \emph{homothetic copy} of $S$ is any set of the form $a+\lambda S$, where $a \in \N^d$ and $\lambda \in \N$. For instance, the homothetic copies of $[n]$ in $\N$ are precisely the APs of length $n$.
\begin{theorem}
  Let $S$ be a finite subset of $\N^d$. Then, whenever $\N^d$ is finitely coloured, there is a monochromatic homothetic copy of $S$.
\end{theorem}
\begin{proof}
  Let $S = \{S(1), \ldots, S(m)\}$. Given a $k$-colouring $c$ of $\N^d$, we induce a $k$-colouring $c'$ of $[m]^n$ where $n$ is large by $c'(x_1, \ldots, x_n) = c(S(x_1)+\ldots +S(x_n))$. We have a monochromatic line $L$ for $c'$ , say with active coordinate set $I$. But now, taking $S(x_1)+\ldots +S(x_n)$ for ecah $x \in L$, we have a monochromatic homothetic copy of $S$ with $\lambda = |I|$.
\end{proof}
There is an alternate proof via product arguments and focusing.


\end{document}
