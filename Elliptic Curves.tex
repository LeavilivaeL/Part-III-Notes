\documentclass[10pt,a4paper]{article}
\usepackage[utf8]{inputenc}
\usepackage{amsmath}
\usepackage{amsfonts}
\usepackage{amssymb}
\usepackage{amsthm}
\usepackage{float}
\usepackage{mathtools}
\usepackage{geometry}[margin=1in]
\usepackage{xspace}
\usepackage{tikz}
\usepackage{mathrsfs}
\usetikzlibrary{shapes, arrows, decorations.pathmorphing, ducks, automata}
\usepackage[parfill]{parskip}
\usepackage{subcaption}
\usepackage{stmaryrd}
\usepackage{marvosym}
\usepackage{dsfont}
\usepackage{pgfplots}
\usepackage{enumitem}
\usepackage{calc}
\usepackage{tikz-cd}
\usepackage{hyperref}

\hypersetup{
    colorlinks,
    citecolor=black,
    filecolor=black,
    linkcolor=black,
    urlcolor=black
}

\newcommand{\st}{\text{ s.t. }}
\newcommand{\contr}{\lightning}
\newcommand{\im}{\mathfrak{i}}
\newcommand{\R}{\mathbb{R}}
\newcommand{\Q}{\mathbb{Q}}
\newcommand{\C}{\mathbb{C}}
\newcommand{\F}{\mathbb{F}}
\newcommand{\K}{\mathbb{K}}
\newcommand{\N}{\mathbb{N}}
\newcommand{\Z}{\mathbb{Z}}
\renewcommand{\P}{\mathbb{P}}
\renewcommand{\H}{\mathds{H}}
\renewcommand{\O}{\mathcal{O}}
\newcommand{\A}{\mathbb{A}}
\newcommand{\D}{\mathbb{D}}
\newcommand{\nequiv}{\not\equiv}
\newcommand{\powset}{\mathcal{P}}
\renewcommand{\th}[1][th]{\textsuperscript{#1}\xspace}
\newcommand{\from}{\leftarrow}
\newcommand{\legendre}[2]{\left(\frac{#1}{#2}\right)}
\newcommand{\ow}{\text{otherwise}}
\newcommand{\imp}[2]{\underline{\textit{#1.}$\implies$\textit{#2.}}}
\let\oldexists\exists
\let\oldforall\forall
\renewcommand{\exists}{\oldexists\;}
\renewcommand{\forall}{\;\oldforall}
\renewcommand{\hat}{\widehat}
\renewcommand{\tilde}{\widetilde}
\newcommand{\one}{\mathds{1}}
\newcommand{\under}{\backslash}
\newcommand{\injection}{\hookrightarrow}
\newcommand{\surjection}{\twoheadrightarrow}
\newcommand{\jacobi}{\legendre}
\newcommand{\floor}[1]{\lfloor #1 \rfloor}
\newcommand{\ceil}[1]{\lceil #1 \rceil}
\newcommand{\cbrt}[1]{\sqrt[3]{#1}}
\renewcommand{\angle}[1]{\langle #1 \rangle}
\newcommand{\dbangle}[1]{\angle{\angle{#1}}}
\newcommand{\wrt}{\text{ w.r.t. }}

\newcommand*\circled[1]{\tikz[baseline=(char.base)]{
      \node[shape=circle,draw,inner sep=2pt] (char) {#1};}
}

\DeclareMathOperator{\ex}{ex}
\DeclareMathOperator{\id}{id}
\DeclareMathOperator{\upper}{Upper}
\DeclareMathOperator{\dom}{dom}
\DeclareMathOperator{\disc}{disc}
\DeclareMathOperator{\charr}{char}
\DeclareMathOperator{\Image}{im}
\DeclareMathOperator{\ord}{ord}
\DeclareMathOperator{\lcm}{lcm}
\DeclareMathOperator{\aut}{Aut}
\DeclareMathOperator{\diag}{diag}
\DeclareMathOperator{\stab}{stab}
\DeclareMathOperator{\trace}{trace}
\DeclareMathOperator{\ecl}{ecl}
\DeclareMathOperator{\Span}{Span}
\DeclareMathOperator{\Gal}{Gal}
\DeclareMathOperator{\Aut}{Aut}
\DeclareMathOperator{\Frob}{Frob}
\let\div\relax
\DeclareMathOperator{\div}{div}
\DeclareMathOperator{\Div}{Div}
\let\Re\relax
\let\Im\relax
\DeclareMathOperator{\Re}{\mathfrak{Re}}
\DeclareMathOperator{\Im}{\mathfrak{Im}}
\DeclareMathOperator{\Frac}{Frac}
\DeclareMathOperator{\Pic}{Pic}

\let\emph\relax
\DeclareTextFontCommand{\emph}{\bfseries\em}

\newtheorem{theorem}{Theorem}[section]
\newtheorem{lemma}[theorem]{Lemma}
\newtheorem{corollary}[theorem]{Corollary}
\newtheorem{proposition}[theorem]{Proposition}
\newtheorem{conjecture}[theorem]{Conjecture}
\newtheorem{definition}[theorem]{Definition}

\definecolor{burgundy}{rgb}{0.5, 0.0, 0.13}

\tikzset{sketch/.style={decorate,
 decoration={random steps, amplitude=1pt, segment length=5pt},
 line join=round, draw=black!80, very thick, fill=#1
}}


\title{Elliptic Curves}
\begin{document}
\maketitle
\section{Fermat's Method of Infinite Descent}
Suppose we have a right-angled triangle $\Delta$ with side lengths $a, b, c$, so that by Pythagoras we have $a^2 + b^2 = c^2$, and $\text{area}(\Delta) = \frac12 ab$.
\begin{definition}
  $\Delta$ is \emph{rational} if $a, b, c \in \Q$, and \emph{primitive} if $a, b, c \in \Z$ coprime.
\end{definition}
\begin{lemma}
  Every primitive triangle is of the form $a = u^2-v^2, b = 2uv, c = u^2+v^2$ for coprime integers $u > v > 0$.
\end{lemma}
\begin{proof}
  If $a, b$ were both odd, then $a^2 + b^2 \equiv 2 \mod 4$, and we have no solutions for $c$. If $a,b$ both even, then they are not coprime. So we may assume $a$ is odd, $b$ is even, $c$ is odd.

  Then $(\frac{b}{2})^2 = \frac{c+a}{2}\frac{c-a}{2}$, and the right hand side is a product of coprime positive integers. So by unique prime factorisation in the integers, $\frac{c+a}{2} = u^2, \frac{c-a}{2} = v^2$ for some coprime integers $u, v$. Rearranging, we have the lemma.
\end{proof}

\begin{definition}
  $D \in \Q_{>0}$ is a \emph{congruent number} if it is the area of a rational triangle.
\end{definition}
Note that, by scaling the triangle, it suffices to consider $D \in \Z_{>0}$ squarefree.

For example, $D = 5,6$ are congruent numbers. $6 = \frac12\cdot 3\cdot 4$, and $3^2+4^2 = 5^2$, and 5 is left as an exercise.

\begin{lemma}
  $D \in \Q_{>0}$ is congruent if and only if $Dy^2 = x^3-x$ for some $x, y \in \Q, y \neq 0$.
\end{lemma}
\begin{proof}
  Lemma \textbf{1.2} shows that $D$ is congruent if and only if $Dw^2 = uv(u^2-v^2)$ for some $u,v,w\in \Q, w\neq 0$.

  Setting $x = \frac{u}{v}, y = \frac{w}{v^2}$ finishes the proof.
\end{proof}

Fermat showed that 1 is not a congruent number.
\begin{theorem}
  There is no solution to
  \begin{align*}
    w^2 = uv(u+v)(u-v)\tag{$\ast$}
  \end{align*} in integers $u,v,w$ with $w \neq 0$.
\end{theorem}
\begin{proof}
  Without loss of generality, $u,v$ are coprime with $u > 0, w> 0$. If $v < 0$ then replace $(u,v,w)$ by $(-v, u, w)$. If $u, v$ are both odd, then replace $(u,v,w)$ by $(\frac{u+v}{2}, \frac{u-v}{2}, \frac{w}{2})$. So we may assume that all of $u, v, u+v, u-v$ are coprime positive integers whose product is a square, and hence are all squares, say $a^2, b^2, c^2, d^2$ respectively, where $a,b,c,d \in \Z_{>0}$.

  Since $u \nequiv v \mod 2$, both $c, d$ are odd. Consider the right angled triangle with side lengths, $\frac{c+d}{2}, \frac{c-d}{2}, a$. This is a primitive triangle, and it has area $\frac{c^2-d^2}{8} = \frac{v}{4} = (\frac{b}{2})^2$.

  Let $w_1 = \frac{b}{2}$. Then lemma \textbf{1.2} gives $w_1^2 = u_1v_1(u_1^2-v_1^2)$ for some $u_1, v_1 \in \Z$, giving a new solution to $(\ast)$. But $4w_1^2 = b^2 = v | w^2$, and so $w_1 \leq \frac12 w$.

  So by Fermat's method of infinite descent, if there were a solution we would have a strictly decreasing infinite sequence of positive integers $\contr$. Hence there is no solution to $(\ast)$.
\end{proof}

\subsection{A Variant for Polynomials}
Here, $K$ is a field with $\charr K \neq 2$. The algebraic closure of $K$ will be $\overline{K}$.
\begin{lemma}
  Let $u, v \in K[t]$ be coprime. If $\alpha u + \beta v$ is a square for four distinct $(\alpha :\beta) \in \P^1$, then $u, v \in K$.
\end{lemma}
\begin{proof}
  Without loss of generality we may assume $K = \overline{K}$, as that doesn't change the degree of polynomials, and every square is still a square.

  Changing coordinates on $\P^1$, we may assume the ratios $\alpha:\beta$ are $(1:0), (0:1), (1:-1), (1:-\lambda)$ for some $\lambda \in K\setminus\{0, 1\}$, with $\mu = \sqrt{\lambda}$.

  Then $u = a^2, v = b^2, u-v = (a+b)(a-b), u-\lambda v = (a+\mu b)(a-\mu b)$ are all squares. They are also coprime, and so by unique factorisation in $K[t]$, $(a+b), (a-b), (a+\mu b), (a-\mu b)$ are all squares.

  But $\max \{\deg a, \deg b\} \leq \frac12 \max \{\deg u, \deg v\}$. So by Fermat's method of infinite descent, we get that the original $u,v \in K$.
\end{proof}

Now we have some important definitions:
\begin{definition}\hspace*{0cm}
  \begin{enumerate}
    \item An \emph{elliptic curve} $E$ over a field $K$ is the projective closure of the affine curve $y^2 = f(x)$ where $f \in K[x]$ is a monic cubic polynomial with distinct roots.
    \item For $L/K$ any field extension, $E(L) = \{ (x,y) \in L^2 : y^2 = f(x)\} \cup \{0\}$. $0$ is called the \emph{point at infinity}.
  \end{enumerate}
\end{definition}
We call the point at infinity $0$ because we will see that $E(L)$ is naturally an abelian group under an operation we will denote by $+$, and $0$ will be the identity for that group. In this course we will study $E(L)$ for $L$ a finite field, a local field, and a number field.

Lemma \textbf{1.4} and theorem \textbf{1.5} together imply that, if $E$ is given by $y^2 = x^3-x$, then $E(\Q) = \{0, (0,0), (\pm 1, 0)\}$, which we will see is the group $C_2\times C_2$.

\begin{corollary}
  Let $E/K$ be an elliptic curve. Then $E(K(t)) = E(K)$.
\end{corollary}
\begin{proof}
  Without loss of generality, $K = \overline{K}$. By a change of coordinates we may assume $E: y^2 = x(x-1)(x-\lambda)$ for some $\lambda \in K\setminus\{0, 1\}$. Suppose $(x, y) \in E(K(t))$. Write $x = \frac{u}{v}$ with $u, v \in K[t]$ coprime. Then $w^2 = uv(u-v)(u-\lambda v)$ for some $w \in K[t]$.

  Unique factorisation in $K[t]$ gives $u, v, u-v, u-\lambda v$ are all squares, and so by lemma \textbf{1.6}, $u, v \in K$, and so $x, y \in K$.
\end{proof}
\end{document}
