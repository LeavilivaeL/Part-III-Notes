\documentclass[10pt,a4paper]{article}
\author{Harry Armitage}

%\usepackage[utf8]{inputenc}
\usepackage{amsmath}
\usepackage{amsfonts}
\usepackage{amssymb}
\usepackage{amsthm}
\usepackage{float}
\usepackage{mathtools}
\usepackage{geometry}[margin=1in]
\usepackage{xspace}
\usepackage{tikz}
\usepackage{mathrsfs}
\usetikzlibrary{shapes, arrows, decorations.pathmorphing, ducks, automata}
\usepackage[parfill]{parskip}
\usepackage{subcaption}
\usepackage{stmaryrd}
\usepackage{marvosym}
\usepackage{dsfont}
\usepackage{pgfplots}
\usepackage{enumitem}
\usepackage{calc}
\usepackage{tikz-cd}
\usepackage{hyperref}
\usepackage[usestackEOL]{stackengine}

\usepackage{fontspec}
\usepackage{newpxtext, newpxmath}
\usepackage{anyfontsize}

\hypersetup{
    colorlinks,
    citecolor=black,
    filecolor=black,
    linkcolor=black,
    urlcolor=black
}

\newcommand{\f}[1]{\mathfrak{#1}}
\newcommand{\p}{\f{p}}

\newcommand{\st}{\text{ s.t. }}
\newcommand{\contr}{\lightning}
\newcommand{\im}{\mathfrak{i}}
\newcommand{\R}{\mathbb{R}}
\newcommand{\Q}{\mathbb{Q}}
\renewcommand{\C}{\mathbb{C}}
\newcommand{\F}{\mathbb{F}}
\newcommand{\K}{\mathbb{K}}
\newcommand{\N}{\mathbb{N}}
\newcommand{\Z}{\mathbb{Z}}
\renewcommand{\P}{\mathbb{P}}
\renewcommand{\H}{\mathds{H}}
\renewcommand{\O}{\mathcal{O}}
\newcommand{\A}{\mathbb{A}}
\newcommand{\D}{\mathbb{D}}
\renewcommand{\G}{\mathbb{G}}
%\newcommand{\nequiv}{\not\equiv}
\newcommand{\powset}{\mathcal{P}}
\renewcommand{\th}[1][th]{\textsuperscript{#1}\xspace}
\newcommand{\from}{\leftarrow}
\newcommand{\legendre}[2]{\left(\frac{#1}{#2}\right)}
\newcommand{\ow}{\text{otherwise}}
\newcommand{\imp}[2]{\underline{\textit{#1.}$\implies$\textit{#2.}}}
\let\oldexists\exists
\let\oldforall\forall
\renewcommand{\exists}{\oldexists\;}
\renewcommand{\forall}{\;\oldforall}
\renewcommand{\hat}{\widehat}
\renewcommand{\tilde}{\widetilde}
\newcommand{\one}{\mathds{1}}
\newcommand{\under}{\backslash}
\newcommand{\injection}{\hookrightarrow}
\newcommand{\surjection}{\twoheadrightarrow}
\newcommand{\isomarrow}{\mathrel{\setstackgap{S}{-0.5pt}\ensurestackMath{\Shortstack{\scriptstyle\sim\\ \longrightarrow}}}}
\newcommand{\jacobi}{\legendre}
\newcommand{\floor}[1]{\lfloor #1 \rfloor}
\newcommand{\ceil}[1]{\lceil #1 \rceil}
\newcommand{\cbrt}[1]{\sqrt[3]{#1}}
\renewcommand{\angle}[1]{\langle #1 \rangle}
\newcommand{\dbangle}[1]{\angle{\angle{#1}}}
\newcommand{\wrt}{\text{ w.r.t. }}
\newcommand{\abs}[1]{\lvert#1\rvert}
\newcommand{\norm}[1]{\lVert#1\rVert}
\newcommand*\circled[1]{\tikz[baseline=(char.base)]{
      \node[shape=circle,draw,inner sep=2pt] (char) {#1};}
}
\renewcommand{\epsilon}{\varepsilon}
\newcommand{\trianglerightneq}{\mathrel{\ooalign{\raisebox{-0.5ex}{\reflectbox{\rotatebox{90}{$\nshortmid$}}}\cr$\triangleright$\cr}\mkern-3mu}}
\newcommand{\triangleleftneq}{\mathrel{\reflectbox{$\trianglerightneq$}}}

\DeclareMathOperator{\ex}{ex}
\DeclareMathOperator{\id}{id}
\DeclareMathOperator{\upper}{Upper}
\DeclareMathOperator{\dom}{dom}
\DeclareMathOperator{\disc}{disc}
\DeclareMathOperator{\charr}{char}
\DeclareMathOperator{\Image}{im}
\DeclareMathOperator{\ord}{ord}
\DeclareMathOperator{\lcm}{lcm}
\DeclareMathOperator{\aut}{Aut}
\DeclareMathOperator{\diag}{diag}
\DeclareMathOperator{\stab}{stab}
\DeclareMathOperator{\trace}{trace}
\DeclareMathOperator{\ecl}{ecl}
\DeclareMathOperator{\Span}{Span}
\DeclareMathOperator{\Gal}{Gal}
\DeclareMathOperator{\Aut}{Aut}
\DeclareMathOperator{\Frob}{Frob}
\DeclareMathOperator{\Det}{Det}
\let\div\relax
\DeclareMathOperator{\div}{div}
\DeclareMathOperator{\Div}{Div}
\let\Re\relax
\let\Im\relax
\DeclareMathOperator{\Re}{\mathfrak{Re}}
\DeclareMathOperator{\Im}{\mathfrak{Im}}
\DeclareMathOperator{\Frac}{Frac}
\DeclareMathOperator{\Pic}{Pic}
\DeclareMathOperator{\ann}{ann}
\DeclareMathOperator{\Ass}{Ass}
\DeclareMathOperator{\intt}{int}
\DeclareMathOperator{\Hom}{Hom}
\DeclareMathOperator{\End}{End}
\DeclareMathOperator{\tr}{tr}
\DeclareMathOperator{\Tr}{Tr}
\DeclareMathOperator{\Spec}{Spec}
\DeclareMathOperator{\height}{ht}
\DeclareMathOperator{\rank}{rank}
\DeclareMathOperator{\Art}{Art}
\DeclareMathOperator{\gr}{gr}
\DeclareMathOperator{\Tor}{Tor}
\DeclareMathOperator{\Ext}{Ext}
\DeclareMathOperator{\coker}{coker}

\let\emph\relax
\DeclareTextFontCommand{\emph}{\bfseries\em}

\newtheorem{theorem}{Theorem}[section]
\newtheorem{lemma}[theorem]{Lemma}
\newtheorem{corollary}[theorem]{Corollary}
\newtheorem{proposition}[theorem]{Proposition}
\newtheorem{conjecture}[theorem]{Conjecture}
\newtheorem{definition}[theorem]{Definition}

\definecolor{burgundy}{rgb}{0.5, 0.0, 0.13}

\tikzset{sketch/.style={decorate,
 decoration={random steps, amplitude=1pt, segment length=5pt},
 line join=round, draw=black!80, very thick, fill=#1
}}


\title{Modular Forms}
\begin{document}
\maketitle
\tableofcontents
\newpage
\setcounter{section}{-1}
\section{Introduction}
\textbf{Notation.} We will write $\H \coloneqq \{\tau \in \C: \Im(\tau) > 0\}$ for the complex upper half plane. This is acted on by two groups: \[GL_2(\R)^+ = \{g \in GL_2(\R): \det(g) > 0\} \geq SL_2(\Z) = \{g \in GL_2(\Z) : \det(g)=1\}\]
\begin{lemma}
  $GL_2(\R)^+$ acts on $\H$ by M\"obius transformations. This action is transitive.
\end{lemma}
\begin{proof}
  Let $\tau \in \H, g = \begin{pmatrix} a&b\\c&d\end{pmatrix} \in GL_2(\R)^+$. We then write $g\tau = \frac{a\tau+b}{c\tau+d}$. This is an action on $\C$ by theory about M\"obius transformations. To see that $g\tau \in \H$, we check:
  \[\Im(g\tau) = \frac{1}{2}(g\tau - \overline{g\tau}) = \det(g) \frac{\Im(\tau)}{|c\tau+d|^2}\]

  Now for transitivity, let $\tau = x+\im y \in \H$. Then $\tau = \begin{pmatrix}y&x\\0&1\end{pmatrix}\im$.
\end{proof}
\begin{definition}
  Let $k \in \Z$, and $f : \H \to \C\cup\{\infty\}$, and let $g \in GL_2(\R)^+$. Then we define $f|_k[g]:\H \to \C\cup\{\infty\}$ by the formula
  \[f|_k[g](\tau) = f(g\tau)\det(g)^{k-1}j(g, \tau)^{-k}\]
  where $j(g, \tau) = c\tau+d$.
\end{definition}
\begin{lemma}
  This defines a right actions of $GL_2(\R)^+$ on the set of functions $f:\H \to \C\cup\{\infty\}$.
\end{lemma}
\begin{proof}
  Suppose $g, h \in GL_2(\R)^+$. We need to show that $f|_k[gh] = (f|_k[g])|_k[h]$.
  \begin{align*}
    RHS(\tau) &= f|_k[g](h\tau)\det(h)^{k-1}j(h, \tau)^{-k}\\
    &= f(gh\tau)\det(g)^{k-1}j(g, h\tau)^{-k}j(h,\tau)^{-k}\det(h)^{k-1}\\
    LHS(\tau) &= f(gh\tau)\det(gh)^{k-1}j(gh,\tau)
  \end{align*}
  So we need to check that $j(g, h\tau)j(h,\tau) = j(gh,\tau)$.

  Note that if $g = \begin{pmatrix}a&b\\c&d\end{pmatrix}$, then $g\begin{pmatrix}\tau\\1\end{pmatrix} = \begin{pmatrix}a\tau+b\\c\tau+d\end{pmatrix} = j(g,\tau)\begin{pmatrix}g\tau\\1\end{pmatrix}$.

  So $gh\begin{pmatrix}\tau\\1\end{pmatrix} = j(gh,\tau)\begin{pmatrix}gh\tau\\1\end{pmatrix} = gj(h,\tau)\begin{pmatrix}h\tau\\1\end{pmatrix} = j(h,\tau)j(g,h\tau)\begin{pmatrix}gh\tau\\1\end{pmatrix}$.
\end{proof}
\begin{definition}
  Let $k \in \Z$, and let $\Gamma \leq SL_2(\Z)$ be a finite index subgroup. Then a meromorphic function $f:\H \to \C \cup\{\infty\}$ is called a weakly modular function of weight $k$ and level $\Gamma$ if it satisfies $\forall \gamma \in \Gamma, f|_k [\gamma] = f$.
\end{definition}

\textbf{Motivating Examples}
\begin{enumerate}
  \item Modular forms were first studied in the context of elliptic functions. Suppose that $E$ is an elliptic curve over $\C$, and let $\omega$ be a non-vanishing holomorphic differential on $E$. Then there's a unique holomorphic isomorphism of Riemann surfaces
  \[\C/\Lambda \xrightarrow[\psi]{\sim} E(\C)\]
  such that $\psi^{\ast}(\omega) = dz$. Here $\Lambda\subset\C$ is a lattice.

  $E$ can be defined by the equation $y^2=x^3-60G_4(\Lambda)x-140G_6(\Lambda)$ where $G_k(\Lambda) = \sum\limits_{\omega \in \Lambda\setminus\{0\}}\omega^{-k}$. This is absolutely convergent provided $k \geq 4$.

  If $\tau \in \H$, then we can write $\Lambda_\tau = \Z\tau \oplus \Z$. This is a lattice, and the functions $G_k(\tau) = G_k(\Lambda \tau)$ are examples of modular forms.

  \item If $f:\H \to \C$ is a modular form, then $f$ has a Fourier expansion $f(\tau) = \sum\limits_{n\geq 0}a_n e^{2\pi\im n\tau/h}$ for some natural number $h$, and complex numbers $a_n$. These Fourier coefficients often carry useful arithmetic information.

  For example, consider $\theta(\tau) = \sum\limits_{n \in \Z}e^{\pi\im n^2\tau}$. If $k \geq 2$ is an even integer, then $\theta^{2k}$ is a modular form of weight $k$. Its Fourier expansion is $\theta^{2k}(\tau) = \sum\limits_{n \geq1}r_{2k}(n)e^{\pi\im n \tau}$ where $r_{2k}(n)$ is the number of ways of writing $n = x_1^2 + \ldots x_{2k}^2$, where $x_i \in \Z$.

  By relating $\theta^{2k}$ to other modular forms with known Fourier series, we can then get information about the numbers $r_{2k}(n)$. For example, $r_4(n) = 8\sum\limits_{d|n, 4\nmid d} d$.

  \item Recall the Riemann zeta function $\zeta(s) = \sum_{n\geq 1}n^{-s}$. This function has some important properties:
  \begin{enumerate}[label=\alph*)]
    \item It has a meromorphic continuation to all of $\C$.
    \item It has a functional equation relating $\zeta(s)$ and $\zeta(1-s)$.
    \item It has a representation as an Euler product $\zeta(s) = \prod_{p}(1-p^{-s})^{-1}$.
  \end{enumerate}
  Any series $L(s) = \sum_{n\geq 1}a_n n^{-s}$ with $a_n \in \C$ which has properties analogous to these is called an $L$-function.

  For example, if $N \in \N$ and $\chi: (\Z/N\Z)^\times \to \C^\times$ is a character, we can define the Dirichlet $L$-function $L(\chi, s) = \sum_{(n,N)=1} \chi(n \mod N)n^{-s}$. These functions can be used to prove Dirichlet's theorem on primes in arithmetic progression.

  Modular forms can be used to construct $L$-functions with these properties. To find the right modular forms, we need to introduce Hecke operators.

  \item The Langlands programme predicts relations between objects ocurring in number theory and modular forms. This includes as a special case the Shimura-Taniyama-Weil conjecture, otherwise known as the modularity theorem. This asserts a bijection between elliptic curves over $\Q$ up to isogeny and certain modular forms, given by ($L$-function of elliptic curve) = ($L$-function of modular form).
\end{enumerate}
\section{Modular forms on $SL_2(\Z)$}
Recall the definition, for $f:\H \to \C, k \in \Z, g \in GL_2(\R)^+$, we have
\[f|_k[g](\tau) = \det(g)^{k-1}f(g\tau)j(g,\tau)^{-k}\]
We said $f$ is \emph{weakly modular of weight k and level $SL_2(\Z)$} if $f$ is meromorphic on $\H$ and, for all $\gamma \in SL_2(\Z)$, $f|_k[\gamma] = f$.

Note that $T = \begin{pmatrix}1&1\\0&1 \end{pmatrix} \in SL_2(\Z)$ satisfies $f|_k[T](\tau) = f(\tau+1)$. So if $f$ is a weakly modular function, then we can define a new function
\[\tilde{f}:\{q \in \C: 0<|q|<1\} \to \C; e^{2\pi\im\tau} \mapsto f(\tau)\]
This function $\tilde{f}$ is meromorphic, since $f$ is.
\begin{definition}
  We say that the weakly modular function $f$ is:
  \begin{itemize}
    \item meromorphic at $\infty$ if $\tilde{f}$ is meromorphic at 0.
    \item holomorphic at $\infty$ if $\tilde{f}$ is holomorphic at 0.
    \item vanishes at $\infty$ if $\tilde{f}$ is holomorphic and vanishes at 0.
  \end{itemize}
\end{definition}
If $f$ is meromorphic at $\infty$ then $\tilde{f}$ has a Laurent expansion $\tilde{f}(q) = \sum_{n \in \Z}a_nq^n$ valid in some region $\{0 <|q|< \epsilon\}$, where $a_n \in \C$ and $a_n = 0$ if $n<0$ and $|n|$ is sufficiently large.

We get a formula $f(\tau) = \sum_{n\in \Z}a_nq^n$ where $q = e^{2\pi\im\tau}$. This is valid in some region $\{\tau \in \H : \Im \tau > R\}$, and is called the $q$-expansion of $f$. Then $f$ is holomorphic at $\infty$ if and only if $a_n = 0$ when $n < 0$, and $f(\infty) = a_0$.
\begin{definition}
  Let $f$ be a weakly modular function of weight $k$ and level $SL_2(\Z)$. We say that $f$ is
  \begin{itemize}
    \item a \emph{modular function} if $f$ is meromorphic at $\infty$.
    \item a \emph{modular form} if $f$ is holomorphic in $\H$ and holomorphic at $\infty$.
    \item a \emph{cuspidal modular form} if $f$ is a modular form vanishing at $\infty$.
  \end{itemize}
  all with weight $k$ and level $SL_2(\Z)$.

  We write $M_k(SL_2(\Z))$ for the $\C$-vector space of modular forms of weight $k$ and level $SL_2(\Z)$. We write $S_k(SL_2(\Z))$ for the subspace of cuspidal modular forms.
\end{definition}
\textbf{Examples.} If $\tau \in \H$, then $\Lambda_\tau = \Z_\tau \oplus \Z$. if $k \in \Z$, then we can define $G_k(\tau) = \sum\limits_{\omega \in \Lambda_\tau\setminus\{0\}}\omega^{-k}$.

If $\gamma = \begin{pmatrix}a&b\\c&d\end{pmatrix} \in SL_2(\Z)$, then $\Lambda_{\gamma \tau} = \Z\left(\frac{a\tau+b}{c\tau+d}\right)\oplus \Z = j(\gamma,\tau)^{-1}\Z(a\tau+b)\oplus \Z(c\tau+d) = j(\gamma,\tau)^{-1}\Lambda_\tau$.

Finally, we find $G_k|_k[\gamma](\tau) = G_k(\gamma\tau)j(\gamma,\tau)^{-k} = \sum\limits_{\omega \in \Lambda_{\gamma\tau}\setminus\{0\}}(\omega j(\gamma,\tau))^{-k} = \sum\limits_{\omega \in \Lambda_\tau\setminus\{0\}} \omega^{-k} = G_k(\tau)$.
\begin{proposition}
  Suppose $k \geq 4$ and $k$ is even. Then $G_k(\tau)$ converges absolutely and uniformly on compact subsets of $\H$. Moreover, $G_k(\tau)$ is holomorphic at $\infty$ and $G_k(\infty) = 2\zeta(k)$. In particular, $G_k \in M_k(SL_2(\Z))$.
\end{proposition}
\textbf{Remark.} We have $-I = \begin{pmatrix} -1&0\\0&-1 \end{pmatrix} \in SL_2(\Z)$ and $f|_k[-I] = f\cdot(-1)^k$, so if $k$ were odd then $f \equiv 0$, and hence $M_k(SL_2(\Z)) = 0$ when $k$ is odd.
\begin{proof}
  Fix $A \geq 1$. Define $\Omega_A = \{\tau \in \H : |\Re(|\tau)|\leq A, \Im(\tau) \geq \frac{1}{A}\}$. We'll show uniform convergence of $G_k$ in $\Omega_A$. Note that if $\tau \in \Omega_A$, then for any $x \in \R$, $|\tau+x| \geq \frac{1}{A}$, and $|\tau+x| \geq \frac{1}{2}|x|$ if $|x|\geq 2A$. Hence $|\tau+x| \geq \sup(1/A, 1/2A^2|x|) \geq \frac{1}{2A^2}\sup(1, |x|)$ for any $x \in \R$.

  If $\tau \in \Omega_A$, then:
  \begin{align*}
    \sum_{(m,n) \in \Z^2\setminus\{0\}} |m\tau+n|^{-k} &= \sum_{(m,n)} |m|^{-k}|\tau+n/m|^{-k} \\
    &\leq \sum_{(m,n)}\frac{|m|^{-k}}{(2A)^{-k}}\sum(1, |n/m|)^{-k}\\
    &= \sum_{(m,n)}(2A)^k \sup(|m|^{-k}, |n|^{-k})\\
    &= \sum_{r \in \N} (2A)^k r^k 8r = (2A)^k8\zeta(k-1)
  \end{align*}
  This shows absolute and uniform convergence.

  To show that $G_k$ is holomorphic at $\infty$ and $G_k(\infty) = 2\zeta(k)$, it's enough to show that
  \[\lim_{\tau\in\Omega_1, \Im\tau\to\infty} G_k(\tau) = 2\zeta(k)\]
  This limit equals $\sum\limits_{(m,n)}\lim\limits_{\Im\tau\to\infty}(m\tau+n)^{-k} = \sum\limits_{n\in \Z\setminus \{0\}}n^{-k} = 2\zeta(k)$, as all terms with $m \neq 0$ vanish.
\end{proof}
$G_k$ is an example of an \emph{Eisenstein series}.
\begin{definition}
  We define the \emph{normalised Eisenstein series} $E_k(\tau) = \frac{1}{2\zeta(k)}G_k(\tau) = 1 + \sum_{n\geq 1}a_nq^n$. We'll see that the $a_n$ are rational numbers of bounded denominators.
\end{definition}
\textbf{Remark.} If $f \in M_k(SL_2(\Z))$ and $g \in M_\ell(SL_2(\Z))$, then $fg \in M_{k+\ell}(SL_2(\Z))$. So $E_4^3, E_6^2 \in M_{12}(SL_2(\Z))$, and $E_4^3(\infty) = E_6^2(\infty)$, so $\Delta = \frac{E_4^3-E_6^2}{1728} \in S_{12}(SL_2(\Z))$. We'll see shortly that $\Delta = \sum_{n\geq 1} b_nq^n$ where $b_1 = 1, b_n \in \Z$ for all $n \geq 1$.

We now study a fundamental domain for the action of $SL_2(\Z)$ on $\H$. We will write $\Gamma(1) = SL_2(\Z)$, and $\overline{\Gamma(1)} = SL_2(\Z)/\angle{-I}$. This will make sense later.

We write \[\mathscr{F} = \{\tau \in \H : -\frac12 \leq \Re\tau \leq \frac12, |\tau|\geq 1\}\] and \[\mathscr{F}' = \{\tau \in \mathscr{F} : \Re \tau < 1/2, |\tau| = 1 \implies \Re\tau \leq 0\}\]
We have elements $T = \begin{pmatrix} 1&1\\0&1\end{pmatrix}$ and $S =\begin{pmatrix}0&-1\\1&0\end{pmatrix}\in \Gamma(1)$.
\begin{proposition}
  $\mathscr{F}$ is a fundamental domain for the action of $\overline{\Gamma(1)}$ on $\H$. More precisely, if $\tau\in\H$ there is $\gamma \in \overline{\Gamma(1)}$ such that $\gamma\tau \in \mathscr{F}$. If $\gamma \tau \in \mathscr{F}^\circ$, then $\gamma$ is unique. Moreover, each $\tau \in \H$ is $\overline{\Gamma(1)}$-conjugate to exactly one element of $\mathscr{F}'$.
\end{proposition}
\begin{proof}
  We first prove that any $\tau \in \H$ is $\overline{\Gamma(1)}$-conjugate to an element of $\mathscr{F}$. We proved earlier that if $\tau \in \H$ and $\gamma \in \Gamma(1)$, then $\Im \gamma(\tau) = \Im(\tau)/|c\tau+d|^2$.

  If $\tau \in \H$, then $\Lambda_\tau = \Z_\tau \oplus \Z$ is a lattice. So as $(c,d) \in \Z^2\setminus\{0\}$, the numbers $|c\tau+d|$ achieve a minimum. Consequently, the numbers $\Im \gamma(\tau)$ for $\gamma \in \Gamma(1)$ achieve a maximum. So \textsc{wlog} we may assume $\Im(\tau) \geq \Im(\gamma \tau)$ for all $\gamma \in \Gamma(1)$. Also \textsc{wlog} we may take $-\frac12 \leq \Re(\tau) \leq \frac12$.

  We then claim that these properties are sufficient for $\tau \in \mathscr{F}$. It is sufficient to show that $|\tau| \geq 1$. We have $\Im(S\tau) = \Im(\tau)/|\tau|^2 \leq \Im(\tau)$, and hence $|\tau|^2 \geq 1$, so we are done.
\end{proof}

\end{document}
