\documentclass[10pt,a4paper]{article}
\author{Harry Armitage}

%\usepackage[utf8]{inputenc}
\usepackage{amsmath}
\usepackage{amsfonts}
\usepackage{amssymb}
\usepackage{amsthm}
\usepackage{float}
\usepackage{mathtools}
\usepackage{geometry}[margin=1in]
\usepackage{xspace}
\usepackage{tikz}
\usepackage{mathrsfs}
\usetikzlibrary{shapes, arrows, decorations.pathmorphing, ducks, automata}
\usepackage[parfill]{parskip}
\usepackage{subcaption}
\usepackage{stmaryrd}
\usepackage{marvosym}
\usepackage{dsfont}
\usepackage{pgfplots}
\usepackage{enumitem}
\usepackage{calc}
\usepackage{tikz-cd}
\usepackage{hyperref}
\usepackage[usestackEOL]{stackengine}

\usepackage{fontspec}
\usepackage{newpxtext, newpxmath}
\usepackage{anyfontsize}

\hypersetup{
    colorlinks,
    citecolor=black,
    filecolor=black,
    linkcolor=black,
    urlcolor=black
}

\newcommand{\f}[1]{\mathfrak{#1}}
\newcommand{\p}{\f{p}}

\newcommand{\st}{\text{ s.t. }}
\newcommand{\contr}{\lightning}
\newcommand{\im}{\mathfrak{i}}
\newcommand{\R}{\mathbb{R}}
\newcommand{\Q}{\mathbb{Q}}
\renewcommand{\C}{\mathbb{C}}
\newcommand{\F}{\mathbb{F}}
\newcommand{\K}{\mathbb{K}}
\newcommand{\N}{\mathbb{N}}
\newcommand{\Z}{\mathbb{Z}}
\renewcommand{\P}{\mathbb{P}}
\renewcommand{\H}{\mathds{H}}
\renewcommand{\O}{\mathcal{O}}
\newcommand{\A}{\mathbb{A}}
\newcommand{\D}{\mathbb{D}}
\renewcommand{\G}{\mathbb{G}}
%\newcommand{\nequiv}{\not\equiv}
\newcommand{\powset}{\mathcal{P}}
\renewcommand{\th}[1][th]{\textsuperscript{#1}\xspace}
\newcommand{\from}{\leftarrow}
\newcommand{\legendre}[2]{\left(\frac{#1}{#2}\right)}
\newcommand{\ow}{\text{otherwise}}
\newcommand{\imp}[2]{\underline{\textit{#1.}$\implies$\textit{#2.}}}
\let\oldexists\exists
\let\oldforall\forall
\renewcommand{\exists}{\oldexists\;}
\renewcommand{\forall}{\;\oldforall}
\renewcommand{\hat}{\widehat}
\renewcommand{\tilde}{\widetilde}
\newcommand{\one}{\mathds{1}}
\newcommand{\under}{\backslash}
\newcommand{\injection}{\hookrightarrow}
\newcommand{\surjection}{\twoheadrightarrow}
\newcommand{\isomarrow}{\mathrel{\setstackgap{S}{-0.5pt}\ensurestackMath{\Shortstack{\scriptstyle\sim\\ \longrightarrow}}}}
\newcommand{\jacobi}{\legendre}
\newcommand{\floor}[1]{\lfloor #1 \rfloor}
\newcommand{\ceil}[1]{\lceil #1 \rceil}
\newcommand{\cbrt}[1]{\sqrt[3]{#1}}
\renewcommand{\angle}[1]{\langle #1 \rangle}
\newcommand{\dbangle}[1]{\angle{\angle{#1}}}
\newcommand{\wrt}{\text{ w.r.t. }}
\newcommand{\abs}[1]{\lvert#1\rvert}
\newcommand{\norm}[1]{\lVert#1\rVert}
\newcommand*\circled[1]{\tikz[baseline=(char.base)]{
      \node[shape=circle,draw,inner sep=2pt] (char) {#1};}
}
\renewcommand{\epsilon}{\varepsilon}
\newcommand{\trianglerightneq}{\mathrel{\ooalign{\raisebox{-0.5ex}{\reflectbox{\rotatebox{90}{$\nshortmid$}}}\cr$\triangleright$\cr}\mkern-3mu}}
\newcommand{\triangleleftneq}{\mathrel{\reflectbox{$\trianglerightneq$}}}

\DeclareMathOperator{\ex}{ex}
\DeclareMathOperator{\id}{id}
\DeclareMathOperator{\upper}{Upper}
\DeclareMathOperator{\dom}{dom}
\DeclareMathOperator{\disc}{disc}
\DeclareMathOperator{\charr}{char}
\DeclareMathOperator{\Image}{im}
\DeclareMathOperator{\ord}{ord}
\DeclareMathOperator{\lcm}{lcm}
\DeclareMathOperator{\aut}{Aut}
\DeclareMathOperator{\diag}{diag}
\DeclareMathOperator{\stab}{stab}
\DeclareMathOperator{\trace}{trace}
\DeclareMathOperator{\ecl}{ecl}
\DeclareMathOperator{\Span}{Span}
\DeclareMathOperator{\Gal}{Gal}
\DeclareMathOperator{\Aut}{Aut}
\DeclareMathOperator{\Frob}{Frob}
\DeclareMathOperator{\Det}{Det}
\let\div\relax
\DeclareMathOperator{\div}{div}
\DeclareMathOperator{\Div}{Div}
\let\Re\relax
\let\Im\relax
\DeclareMathOperator{\Re}{\mathfrak{Re}}
\DeclareMathOperator{\Im}{\mathfrak{Im}}
\DeclareMathOperator{\Frac}{Frac}
\DeclareMathOperator{\Pic}{Pic}
\DeclareMathOperator{\ann}{ann}
\DeclareMathOperator{\Ass}{Ass}
\DeclareMathOperator{\intt}{int}
\DeclareMathOperator{\Hom}{Hom}
\DeclareMathOperator{\End}{End}
\DeclareMathOperator{\tr}{tr}
\DeclareMathOperator{\Tr}{Tr}
\DeclareMathOperator{\Spec}{Spec}
\DeclareMathOperator{\height}{ht}
\DeclareMathOperator{\rank}{rank}
\DeclareMathOperator{\Art}{Art}
\DeclareMathOperator{\gr}{gr}
\DeclareMathOperator{\Tor}{Tor}
\DeclareMathOperator{\Ext}{Ext}
\DeclareMathOperator{\coker}{coker}

\let\emph\relax
\DeclareTextFontCommand{\emph}{\bfseries\em}

\newtheorem{theorem}{Theorem}[section]
\newtheorem{lemma}[theorem]{Lemma}
\newtheorem{corollary}[theorem]{Corollary}
\newtheorem{proposition}[theorem]{Proposition}
\newtheorem{conjecture}[theorem]{Conjecture}
\newtheorem{definition}[theorem]{Definition}

\definecolor{burgundy}{rgb}{0.5, 0.0, 0.13}

\tikzset{sketch/.style={decorate,
 decoration={random steps, amplitude=1pt, segment length=5pt},
 line join=round, draw=black!80, very thick, fill=#1
}}


\title{Commutative Algebra}
\begin{document}
\maketitle
\setcounter{section}{-1}
\section{Introduction}
Commutative Algebra is the study of commutative rings and the spaces on which those rings act, namely modules. It was developed from two key sources: algebraic geometry, and algebraic number theory.

In algebraic geometry we are focused on polynomial rings over a field $k$, whilst in number theory we are focused on $\Z$, the ring of rational integers. Much of this work was done by Grothedieck, but the subject goes back much further, at least to Hilbert who wrote a series of papers on polynomial invariant theory in the late nineteenth century.

As an example, take $\Sigma_n$, the symmetric group on the set $\{1,2,\ldots,n\}$. $\Sigma_n$ acts on $k[x_1, \ldots, x_n]$ by permuting the variables, so that $(\sigma f)(x_1, \ldots, x_n) = f(x_{\sigma^{-1}(1)}, \ldots, x_{\sigma^{-1}(n)})$. $\sigma_n$ acts here via ring automorphisms, and it is then natural to consider the \emph{ring of invariants}, given by $\{f\in k[\mathbf{x}]: \sigma f = f\; \forall \sigma \in \Sigma_n] \coloneqq S$. $S$ is a ring, \emph{the ring of symmetric polynomials}. We can consider the elementary symmetric functions, which are:
\begin{align*}
e_1(x_1, \ldots, x_n) &= x_1 + \ldots + x_n\\
e_2(x_1, \ldots, x_n) &= \sum_{i<j} x_ix_j\\
&\vdots \\
e_n(x_1, \ldots, x_n) &= x_1\ldots x_n
\end{align*}

In fact, $S$ is generated as a ring by these $e_i$, and there are canonical maps $k[y_1, \ldots, y_n] \to S$ such that $Y_i \mapsto e_i$, which is a ring isomorphism.

Hilbert showed that $S$ is finitely generated, and moreover for many other groups, not just symmetric groups.

Along the way, he proved four very deep theorems:
\begin{itemize}
\item Basis theorem
\item Nullstellensatz
\item The polynomial nature of the Hilbert function (leading to the beginnings of dimension theory)
\item The syzygy theorem (leading to the beginnings of homological theory of polynomial rings)
\end{itemize}

In 1921 Emmy Noether extracted the key property that made the basis theorem, namely that a commutative ring is \emph{noetherian} if every ideal is finitely generated (there are several equivalent definitions).

\begin{theorem}[Hilbert's Basis Theorem]
If $R$ is a commutative noetherian ring, then $R[x]$ is also noetherian.
\end{theorem}
\begin{corollary}
If $k$ is a field, then $k[x_1, \ldots, x_n]$ is noetherian.
\end{corollary}

Noether developed a theory of ideals for noetherian rings, for example the existence of primary decomposition, which generalises factorisation into primes in noetherian rings.

\subsection*{Link between Commutative Algebra and Algebraic Geometry}
The starting point for this link is the \emph{fundamental theorem of algebra}, which says that $f \in \C[x]$ is determined up to scalar multiples by its zeros up to multiplicity. Given $f \in \C[x_1, \ldots, x_n]$, there is a polynomial function $\C^n \to \C$ given by $(a_1, \ldots, a_n) \mapsto f(a_1, \ldots, a_n)$.

Different polynomials will yield different functions, and so $\C[x_1, \ldots, x_n]$ can be viewed as a ring of polynomial functions on complex affine n-space.

More specifically, given $I \subseteq \C[x_1, \ldots, x_n]$, we can define the \emph{set of common zeros}, $Z(I) = \{(a_1, \ldots, \_n) \in \C^n : f(a_1, \ldots, a_n) = 0 \; \forall f \in I\}$, called an \emph{(affine) algebraic set}.

\underline{Remarks:}
\begin{itemize}
\item One can replace $I$ by the ideal generated by $I$, and you get the same algebraic set. Similarly, replacing an ideal by a generating set of the ideal leaves the algebraic set. The basis theorem asserts that any algebraic set is the set of common zeros of some \emph{finite} set of polynomials.

\item $\bigcap_j Z(I_j) = Z(\bigcup_j I_j), \bigcup_{j=1}^n Z(i_j) = Z(\prod_{j=1}^n I_j)$, for ideal $I_j$. If we define a topology on $\C^n$ by calling these algebraic sets the closed sets, we get the \emph{Zariski toplogy}, which is a rather coarser topology on $\C^n$ than the usual topology.

\item For $S \subseteq \C^n$, we can define $I(S) = \{f \in \C[x_1, \ldots, x^n] : f(a_1, \ldots, a_n)=0\;\forall (a_1, \ldots, a_n) \in S\}$. This is an \emph{ideal} of $\C[x_1, \ldots, x_n]$, and it is \emph{radical}, i.e. $f^r \in I(s) \implies f \in I(S)$. The Nullstellensatz is a family of results asserting that the correspondence
\begin{align*}
I &\mapsto Z(I)\\
I(S) &\mapsfrom S
\end{align*}
gives a bijection between the radical ideals in $\C[x_1, \ldots, x_n]$ and the algebraic subsets of $\C^n$. In particular, the maximal ideals of $\C[x_1, \ldots, x_n]$ correspond to points in $\C^n$
\end{itemize}

\subsection*{Dimension}
A large portion of the course deals with the dimension of rings. We can define it in three main ways:
\begin{itemize}
\item The maximal length of a chain of prime ideals.
\item In a geometric context in terms of growth rates.
\item The transcendence degree of a field of fractions.
\end{itemize}
For commutative rings, all three give the same answer. There is in fact a fourth method, using homological algebra, which in the case of ``nice" noetherian rings also gives the same answer.

Most of this theory dates back to 1920-1950. Rings of dimension 0 are called \emph{artinian} rings, and in dimension 1 there are special properties which are important in number theory, particularly in the study of algebraic curves.

\section{Noetherian Rings: Definitions and Examples}
Throughout this section, $R$ is a commutative ring with a 1.

\begin{lemma}
Let $M$ be a (left) $R$-module. The following are equivalent:
\begin{enumerate}
\item All submodules of $M$ (including $M$ itself) are finitely generated.
\item The ascending chain condition (ACC) holds: there are no strictly increasing infinite chains of submodules.
\item The maximum condition of submodules holds: and nonempty set $S$ of submodules of $M$ has a maximal element $L$, i.e. $L \subseteq L', L' \in S \implies L = L'$.
\end{enumerate}
\end{lemma}
\begin{proof}\hspace*{0cm}\\
\imp{1}{2} Suppose there is a strictly increasing chain $N_1 \subsetneq N_2 \subsetneq \ldots$, and let $N = \bigcup_{i=1}^\infty N_i$. By \textit{1} $N$ is finitely generated, say by $m_1, \ldots, m_r$. Each $m_i$ lies in some $N_{n_i}$. Then let $n = \max_i n_i$, so that $m_i \in N_n$. Then $N_n = M$, contradicting strict ascent.

\imp{2}{3} Assume ACC. Pick $M_1 \in S$. If it is the maximal member then we're done. If not, there is $M_2 \supsetneq M_1$. If $M_2$ is maximal, then we're done, otherwise there is some $M_3 \supsetneq M_2$, and so on. By ACC this process terminates, and we get a maximal element.

\imp{3}{1} Let $N \triangleleft M$, and let $S$ be the collection of all finitely generated submodules of $N$. Then $S \neq \emptyset$ since it contains the 0 submodule. So $S$ contains a maximal member, say $L$. We then claim $N = L$. If $x \in N$ then $L+Rx \in S$, and by maximality of $L$, $x \in L$.
\end{proof}

\begin{definition}
An $R$-module satisfying \textit{1, 2, 3} is \emph{noetherian}.
\end{definition}

\begin{lemma}
Let $N \triangleleft M$. Then $M$ is noetherian if and only if $N$ and $M/N$ are noetherian.
\end{lemma}
\begin{proof}\hspace*{0cm}\\
\underline{$\implies$} Let $M$ be noetherian, so that all its submodules are finitely generated. This property is inherited by $N$. Also, the submodules of $M/N$ are all of the form $Q/N$ with $Q \triangleleft M$ containing $N$. If $M$ is noetherian, then $Q$ is finitely generated, say by $x_1, \ldots, x_r$. Then $x_1 + N, \ldots, x_r+ N$ generates $Q/N$.

\underline{$\impliedby$} Let $N, M/N$ be noetherian, and let $L_1 \subset L_2 \subset L_3 \subset \ldots$ be a strictly increasing chain of submodules of $M$. Set $Q_i / N = (L_i + N)/N$, and $N_i = L_i \cap N$. These give ascending chains of submodules of $M/N$ and $N$ respectively. By ACC there are $r, s$ with $Q_i/N = Q_r/N$ for $i\geq r$, $N_i = N_s$ for $i\geq s$. Let $k = \max\{r, s\}$. Then we claim $L_i = L_k$ for $i \geq k$. Pick $\ell \in L_i$, $i \geq k$. Then $\ell + N \in Q_k/N$, and so there is some $\ell' \in L_k$ such that $\ell-\ell' \in N \cap L_i = N\cap L_k$. So $\ell \in L_k$, and the claim is proved. Hence our original ascending chain was not strictly increasing, $\contr$.
\end{proof}
\begin{lemma}
\begin{enumerate}
\item If $M, N$ are $R$-modules, then $M \oplus N$ is noetherian iff $M$ and $N$ are noetherian.
\item If $M_1, \ldots, M_n$ are $R$-modules then $M_1 \oplus \ldots \oplus M_n$ is noetherian iff each $M_i$ is noetherian.
\item If $M$ is noetherian then every homomorphic image of $M$ is noetherian.
\item Suppose $M$ can be expressed as a sum of finitely many submodules (not necessarily as a direct sum) $M = M_1 + \ldots + M_n$. Then $M$ is noetherian iff each $M_i$ is.
\end{enumerate}
\end{lemma}
\begin{proof}
\begin{enumerate}
\item $M \cong N/N$, so this follows by \textbf{1.3}.
\item Apply \textit{1} and induction on $n$.
\item If $\theta: M \to N$ then $\Image \theta \cong M/\ker\theta$, so apply \textbf{1.3}.
\item The forwards direction follows as $M_i \triangleleft M$. For the reverse, there is a map from $M_1 \oplus \ldots \oplus M_n \to M$, $(m_1, \ldots, m_n) \mapsto m_1+\ldots+m_n$, and then apply \textit{2} and \textit{3}.
\end{enumerate}
\end{proof}

\begin{definition}
A ring $R$ is \emph{noetherian} if it is noetherian as a (left) $R$-module
\end{definition}

\underline{Remark:}  Submodules of $R$ as an $R$-module are the same as ideals of $R$ as a ring, and so the ACC for modules gives us the ACC for ideals.

\begin{lemma}
Let $R$ be a noetherian ring. Then any finitely generated $R$-module $M$ is noetherian.
\end{lemma}
\end{document}
